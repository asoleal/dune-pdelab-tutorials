\documentclass[a4paper,12pt]{article}

\usepackage[utf8]{inputenc}
\usepackage[T1]{fontenc}
\usepackage[a4paper,total={150mm,240mm}]{geometry}
\usepackage{amsmath}
\usepackage{amsfonts}
\usepackage{amsthm}
\usepackage{amscd}
\usepackage{grffile}
\usepackage{tikz} 
\usepackage{eurosym} 
\usepackage{graphicx}
\usepackage{color}
\usepackage{listings}
\lstset{language=C++, basicstyle=\ttfamily, 
  keywordstyle=\color{black}\bfseries, tabsize=4, stringstyle=\ttfamily,
  commentstyle=\itshape, extendedchars=true, escapeinside={/*@}{@*/}}
\usepackage{paralist}
\usepackage{curves}
\usepackage{calc}
\usepackage{picinpar}
\usepackage{enumerate}
\usepackage{algpseudocode}
\usepackage{bm}
\usepackage{multibib}
\usepackage{hyperref}
\usepackage{textcase}
\usepackage{nicefrac}

\definecolor{listingbg}{gray}{0.95}

\title{DUNE PDELab Tutorial 02 \\ Cell-Centered Finite Volume Method}
\author{DUNE/PDELab Team}
\date{\today}

\begin{document}

\maketitle
\tableofcontents
\clearpage

\section{Introduction}

This tutorial solves the same partial differential equation (PDE) as tutorial 01, namely
a nonlinear Poisson equation, with the following differences:
\begin{enumerate}[1)]
\item Implements a cell-centered finite volume method with two-point flux
approximation as an example of a non-conforming scheme.
\item Implements \textit{all} possible methods of a local operator.
\end{enumerate}

\subsection*{Depends On} Tutorial 00 and 01.

\section{PDE Problem}

Consider the following nonlinear Poisson equation (the same as in tutorial 01) with
Dirichlet and Neumann boundary conditions:
\begin{subequations} \label{eq:ProblemStrong}
\begin{align}
-\Delta u + q(u) &= f &&\text{in $\Omega$},\\
u &= g &&\text{on $\Gamma_D\subseteq\partial\Omega$},\\
-\nabla u\cdot \nu &= j &&\text{on $\Gamma_N=\partial\Omega\setminus\Gamma_D$}.
\end{align}
\end{subequations}
$\Omega\subset\mathbb{R}^d$ is a domain, $q:\mathbb{R}\to\mathbb{R}$ is a given, possibly
nonlinear function and $f: \Omega\to\mathbb{R}$ is the source term and
$\nu$ denotes the unit outer normal to the domain.

\subsection{Outlook}

Here are some suggestions how to test and modify this example:
\begin{itemize}
\item Compare the convergence of Newton's method in tutorial 01 and 02. Does the
exact Jacobian in combination with a different discretization scheme make any difference?
\end{itemize}

% bibtex bibliography
\bibliographystyle{plain}
\bibliography{tutorial02.bib}

\end{document}
