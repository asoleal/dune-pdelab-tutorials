\documentclass[a4paper,12pt]{article}

\usepackage[utf8]{inputenc}
\usepackage[T1]{fontenc}
\usepackage[a4paper,total={150mm,240mm}]{geometry}
\usepackage{amsmath}
\usepackage{amsfonts}
\usepackage{amsthm}
\usepackage{amscd}
\usepackage{grffile}
\usepackage{tikz} 
\usepackage{eurosym} 
\usepackage{graphicx}
\usepackage{color}
\usepackage{listings}
\lstset{language=C++, basicstyle=\ttfamily, 
  keywordstyle=\color{black}\bfseries, tabsize=4, stringstyle=\ttfamily,
  commentstyle=\itshape, extendedchars=true, escapeinside={/*@}{@*/}}
\usepackage{paralist}
\usepackage{curves}
\usepackage{calc}
\usepackage{picinpar}
\usepackage{enumerate}
\usepackage{algpseudocode}
\usepackage{bm}
\usepackage{multibib}
\usepackage{hyperref}
\usepackage{textcase}
\usepackage{nicefrac}

\definecolor{listingbg}{gray}{0.95}

\title{DUNE PDELab Tutorial 01 \\ 
Conforming Finite Element Method for a Nonlinear Poisson Equation}
\author{Peter Bastian\\
  Universität Heidelberg, \\
  Interdisziplinäres Zentrum für Wissenschaftliches Rechnen\\
  Im Neuenheimer Feld 368, D-69120 Heidelberg\\
  \url{Peter.Bastian@iwr.uni-heidelberg.de}
}
\date{\today}

\begin{document}

\maketitle
\tableofcontents
\clearpage

\section{Introduction}

In this tutorial we extend tutorial 00 in various ways:
\begin{enumerate}[1)]
\item Solve a nonlinear stationary partial PDE.
\item Use conforming finite element spaces of arbitrary order.
\item Use different types of (conforming) meshes (simplical, cubed and mixed).
\end{enumerate}
Combined with the fact that the implementation works in any dimension
(note: it is not claimed to be efficient in high dimension) this comprises
already a relatively large space of different methods. Moreover, the 
finite element method developed in this tutorial will serve as a 
building block for instationary problems, adaptive mesh refinement and
parallel solution in subsequent tutorials.

\subsection*{Depends On} 

This tutorial depends on tutorial 00 which discusses piecewise linear elements on simplicial
elements. It is assumed that you have worked through tutorial 00 before.

\section{Problem Formulation}

Here we consider the following nonlinear Poisson equation with
Dirichlet and Neumann boundary conditions:
\begin{align*}
-\Delta u + q(u) &= f &&\text{in $\Omega$},\\
u &= g &&\text{on $\Gamma_D\subseteq\partial\Omega$},\\
-\nabla u\cdot \nu &= j &&\text{on $\Gamma_N=\partial\Omega\setminus\Gamma_D$}.
\end{align*}
$\Omega\subset\mathbb{R}^d$ is a domain, $q:\mathbb{R}\to\mathbb{R}$ is a given, possibly
nonlinear function and $f: \Omega\to\mathbb{R}$ is the source term and
$\nu$ denotes the unit outer normal to the domain.

The weak formulation of this problem is derived by multiplication with an appropriate
test function and integrating by parts. This results in the abstract problem:
\begin{equation}
\text{Find $u\in U$ s.t.:} \quad r^{\text{NLP}}(u,v)=0 \quad \forall v\in V,
\label{Eq:BasicBuildingBlock}
\end{equation}
with the continuous residual form
\begin{equation*}
r^{\text{NLP}}(u,v) = \int_\Omega \nabla u \cdot \nabla v + (q(u)-f)v\,dx + \int_{\Gamma_N} jv\,ds
\label{eq:ResidualForm}
\end{equation*}
and the function spaces 
$U= \{v\in H^1(\Omega) \,:\, \text{``$v=g$'' on $\Gamma_D$}\}$
and $V= \{v\in H^1(\Omega) \,:\, \text{``$v=0$'' on $\Gamma_D$}\}$. 
We assume that $q$ is such that this problem has a unique solution.

\section{Finite Element Method}

The finite element method replaces the function spaces $U$ and $V$ by
finite dimensional approximations defined on a finite element mesh.
Before describing exactly how these spaces are constructed let us
explore the consequences of this.

\subsection{Algebraic Problem}

A finite-dimensional function space is spanned by a basis. So, assume that
\begin{equation*}
U_h=\text{span}\{\phi_1,\ldots,\phi_n\}, \quad V_h=\text{span}\{\psi_1,\ldots,\psi_n\}
\end{equation*}
are corresponding sets of basis functions for $U_h$ and $V_h$.
Expanding the solution $u_h=\sum_{j=1}^n (z)_j\phi_j$ in the basis and
hereby introducing the coefficient vector $z\in\mathbb{R}^n$ we can
reformulate the problem as
\begin{align*}
\text{Find $u_h\in U_h$ s.t.:} && r(u_h,v)&=0 && \forall v\in V_h\\
\Leftrightarrow{} && r\left(\sum_{j=1}^n (z)_j\phi_j,\psi_i\right) &= 0 &&\forall i=1,\ldots,m\\
\Leftrightarrow{} && R(z) = 0,
\end{align*}
where $R: \mathbb{R}^n \to \mathbb{R}^m$ given by 
$R_i(z) = r_h\left(\sum_{j=1}^n (z)_j\phi_j,\psi_i\right)$ is a nonlinear, vector-valued function.

The solution of the nonlinear algebraic equation $R(z)=0$ is typically computed
in an iterative fashion using e.g. a fixed-point iteration of the form
\begin{equation}
z^{(k+1)} = G(z^{(k)}) = z^{(k)} - \lambda^{k} W(z^{(k)}) R(z^{(k)}) .
\end{equation}
Here $\lambda^{k}$ is a damping factor
and $W(z^{(k)})$ is a preconditioner matrix, e.g. in Newton's method one
has 
\begin{equation*}
W(z^{(k)}) = (J(z^{(k)}))^{-1} \quad \text{where $(J(z^{(k)}))_{i,j} = \frac{\partial R_i}{\partial z_j}
(z^{(k)})$}
\end{equation*}
(we now assumed that $n=m$ and that the Jacobian $J(z^{(k)})$ is invertible).
Newton's method requires the solution of the linear system $J(z^{(k)}) w = R(z^{(k)})$ in each
step which could be done using either direct or iterative methods.
The implementation of  Newton's method requires the following
algorithmic building blocks:
\begin{enumerate}[i)]
\item residual evaluation $R(z)$,
\item Jacobian evaluation $J(z)$ (or an approximation of it),
\item matrix-free Jacobian application $J(z) w$ (or an approximation).
\end{enumerate}
Only one of the methods i) and ii) is required depending on the chosen
solution procedure.

\subsection{Finite Element Space}

The detailed construction of the basis functions $\phi_j$ involves the finite element mesh.
Wrapping up the notation from tutorial 00, a finite element mesh consists of
\begin{enumerate}[i)]
\item A set of vertices $\mathcal{X}_h = \{x_1,\ldots,x_N\}$ and
elements $\mathcal{T}_h = \{T_1, \ldots, T_M\}$. Elements are closed and connected sets of points
with non-intersecting interior partitioning the domain $\Omega$.
\item A partitioning of the vertex index set $\mathcal{I}_h=\{1,\ldots,N\}$
into indices of interior and boundary vertices
\begin{equation*}
\mathcal{I}_h = \mathcal{I}_h^{int}\cup\mathcal{I}_h^{\partial\Omega},
\quad \mathcal{I}_h^{int} = \{i\in \mathcal{I}_h\,:\, x_i\in\Omega\},
\quad \mathcal{I}_h^{\partial\Omega} = \{i\in \mathcal{I}_h\,:\, x_i\in\partial\Omega\}.
\end{equation*}
\item For every element $T\in\mathcal{T}_h$ a local-to-global map 
$$g_T:\{0,\ldots,n_T-1\}\to\mathcal{I}_h$$ 
associating a local number of a corner of element $T$ with a global vertex number.
$n_T$ is the number of corners of element $T$.
\item For every element $T\in\mathcal{T}_h$ an element transformation map
$$\mu_T : \hat T \to T$$
mapping the corresponding reference element to $T$. The element transformation
map need not be affine but is assumed to be suffiently differentiable with invertible
Jacobian as well as consistent in the sense 
$\forall i\in\{0,\ldots,n_T-1\} : \mu_T(\hat x_i) = x_{g_T(i)}$.
\end{enumerate}

The finite element space of degree $k$ in dimension $d$ on the mesh
$\mathcal{T}_h$ is given by
\begin{equation}
V_h^{k,d}(\mathcal{T}_h) = \left\{ v\in C^0(\overline{\Omega}) \,:\, 
\forall T\in\mathcal{T}_h : v|_T = \mu_T \circ p_T \wedge p_T\in\mathbb{P}_T^{k,d}\right\}
\label{eq:Vh}
\end{equation}
with the appropriate space of multivariate polynomials of degree $k$ in dimension $d$
depending on the type of element $T$:
\begin{equation}
\mathbb{P}_T^{k,d} = \left\{\begin{array}{ll}
\left\{ p\,:\, p(x_1,\ldots,x_d) = \sum\limits_{0\leq\|\alpha\|_1\leq k} c_\alpha
x_1^{\alpha_1}\cdot\ldots\cdot x_d^{\alpha_d}\right\} & \text{$\hat T = \hat S$ (simplex)}, \\
\left\{ p\,:\, p(x_1,\ldots,x_d) = \sum\limits_{0\leq\|\alpha\|_\infty\leq k} c_\alpha
x_1^{\alpha_1}\cdot\ldots\cdot x_d^{\alpha_d}\right\} & \text{$\hat T = \hat C$ (cube)} .
\end{array}\right .
\end{equation}
Note that in dimension $1$ there is no difference between cube and simplex.
In dimension $2$ triangular and quadrilateral elements may be mixed. In 
dimension $3$, however, tetrahedral and hexahedral elements may not be mixed
without introducing additional elements such as prisms.
The dimension of  $\mathbb{P}_T^{k,d}$ is
\begin{equation*}
n_{\hat C}^{k,d} = (k+1)^d
\end{equation*}
in the case of a cube reference element and
\begin{equation*}
n_{\hat S}^{k,d} = \left\{ \begin{array}{ll}
1 & k=0 \vee d=0\\
\sum_{i=0}^k n_{\hat S}^{i,d-1} &\text{else}
\end{array}\right .
\end{equation*}
in the case of a simplex reference element.

\subsubsection*{Local Lagrange Basis}

\eqref{eq:Vh} defines the finite element space without reference to a basis.
For the implementation in the computer a basis is needed. We now generalize
the construction of the Lagrange basis functions to the general space $V_h^{k,d}(\mathcal{T}_h)$.
To that end, the reference element $\hat T$ is equipped with Lagrange  
points 
$$L_{\hat T} = \left\{ \hat x^{\hat T}_0,\ldots,\hat x^{\hat T}_{n_{\hat T}^{k,d}-1} \right\}$$
and Lagrange polynomials 
$$P_{\hat T} = \left\{ p^{\hat T}_0,\ldots,p^{\hat T}_{n_{\hat T}^{k,d}-1}\right\}$$
such that $$p^{\hat T}_i(\hat x^{\hat T}_j) = \delta_{i,j}.$$
Global continuity is then ensured by carefully placing $n_{\hat T}^{k,d-c}$ of these
points on each face of codimension $c$ in the reference element. 

\subsubsection*{Global Lagrange Basis}

The local-to-global map $g_T$ is now
extended (note that the corners of the reference element
are the Lagrange points for degree $k=1$) to the full set of Lagrange points
such that
$$ g_T(i) = g_{T'}(i')\quad\Leftrightarrow\quad 
\mu_{T}(x^{\hat T}_i)=\mu_{T'}(x^{\hat T'}_{i'}) .$$
Conversely, the set
$$C(i) = \{(T,m)\in\mathcal{T}_h\times\mathbb{N} \,:\, g(T,m)=i\}$$
gives all element and local number pairs that are mapped to the global
degree of freedom number $i$.

The global Lagrange basis functions spanning $V_h^{k,d}(\mathcal{T}_h)$ are then defined by
\begin{equation*}
\phi_i(x) = \left\{\begin{array}{ll}
p^{\hat T}_m(\mu_T^{-1}(x)) & x\in T \wedge (T,m)\in C(i) \\
0 & \text{else}
\end{array}\right. \quad 0\leq i < N_h^{k,d} = \text{dim} V_h^{k,d}(\mathcal{T}_h) .
\end{equation*}

\subsection{Incorporation of Boundary Conditions}


\subsection{Element-wise Computations}

The residual form \eqref{eq:ResidualForm} can be readily decomposed into
elementwise contributions:
\begin{equation*}
r^{\text{NLP}}\left(u,v\right) =  
\sum_{T\in\mathcal{T}_h} \alpha_T^V(u,v) 
  + \sum_{T\in\mathcal{T}_h} \lambda_T^V(v)
 + \sum_{F\in\mathcal{F}_h^{\partial\Omega}}\lambda_F^B(v)
\end{equation*}
with
\begin{align*}
\alpha_T^V(u,v) &= \int_T \nabla u \cdot \nabla v + q(u) v \,dx, &
\lambda_T^V(v) &= - \int_T f v \,dx, &
\lambda_F^B(v) &= \int_{F\cap\Gamma_N} j v\,ds.
\end{align*}
Here $\mathcal{F}_h^{\partial\Omega}$ is the set of intersections of
elements with the domain boundary, i.e. $F=T_F^-\cap\partial\Omega$
where $T_F^-$ is the element associated with $F$. 
The element-wise computations can be classified on the one hand as volume
integrals (superscript $V$), boundary integrals (superscript $B$) and
skeleton integrals (superscript $S$, to be shown later) and on the
other hand as integrals depending on trial and test functions ($\alpha$-terms)
and integrals depending only on test functions ($\lambda$-terms). Here we need
three of these six possible combinations.

The three terms can now be evaluated using the techniques introduced in 
turorial 00 with the small extension that for general maps $\mu_T$ we
have 
$$\nabla w(\mu_T(\hat x)) = J_{\mu,T}^{-1}(\hat x) \hat\nabla \hat w (\hat x)$$
with $J_{\mu,T}(\hat x)$ the Jacobian of $\mu_T$ at point $\hat x$.

\subsubsection*{$\lambda$ Volume Term}

For any $(T,m)\in C(i)$ we obtain
\begin{equation*}
\begin{split}
\lambda_T^V(\phi_i) &= - \int_T f \phi_i \,dx = 
- \int_{\hat T} f(\mu_T(\hat x)) p_m^{\hat T}(\hat x) |\text{det} J_{\mu,T}(\hat x)|\, d\hat x .
\end{split}
\end{equation*}
This integral on the reference element is then computed by employing
numerical integration of appropriate order.
The evaluation for all test functions with support on element $T$ may be collected in
a vector 
\begin{equation*}
(\mathcal{L}_T^V)_m = - \int_{\hat T} f(\mu_T(\hat x)) p_m^{\hat T}(\hat x) 
|\text{det} J_{\mu,T}(\hat x)|\, d\hat x.
\end{equation*}

\subsubsection*{$\lambda$ Boundary Term}

For $F\in\mathcal{F}_h^{\partial\Omega}$ with $F\cap\Gamma_N\neq\emptyset$
and $(T_F^-,m)\in C(i)$ we obtain
\begin{equation*}
\begin{split}
\lambda_T^B(\phi_i) &= \int_{F} j v\,ds = 
\int_{\hat F} j(\mu_F(\hat x)) p_m^{\hat T}(\eta_F(\hat x)) 
\sqrt{|\text{det} J^T_{\mu,T}(\hat x)J_{\mu,T}(\hat x)|} \,ds
\end{split}
\end{equation*}
Because integration is over a face of codimension 1 now, two mappings are
involved. The map $\mu_F$ maps the reference element $\hat F$ of $F$ into
global coordinates while the map $\eta_F$ maps $\hat F$ into the referece
element $\hat T$ of $T$. Also the integration element has to redefined accordingly.
Again, all contributions of the face $F$ can e collected in a vector:
\begin{equation*}
(\mathcal{L}_T^V)_m = 
\int_{\hat F} j(\mu_F(\hat x)) p_m^{\hat T}(\eta_F(\hat x)) 
\sqrt{|\text{det} J^T_{\mu,T}(\hat x)J_{\mu,T}(\hat x)|} \,ds .
\end{equation*}

\subsubsection*{$\alpha$ Volume Term}

For any $(T,m)\in C(i)$ we get
\begin{equation*}
\begin{split}
\alpha_T^V(u_h,\phi_i) &= \int_T \nabla u \cdot \nabla \phi_i + q(u) \phi_i \,dx,
= \int_T \sum_j (z)_j \left(\nabla \phi_j \cdot \nabla \phi_i \right) 
+ q\left( \sum_j (z)_j \phi_j \right) \phi_i \,dx,\\
&= \int_{\hat T} \sum_{n} (z)_{g_T(n)} (J_{\mu,T}^{-1}(\hat x) \hat\nabla p_n^{\hat T}(\hat x) )
\cdot (J_{\mu,T}^{-1}(\hat x) \hat\nabla p_m^{\hat T}(\hat x) ) \\
&\hspace{40mm}+ q\left( \sum_n (z)_{g_T(n)} p_n^{\hat T}(\hat x) \right) p_m^{\hat T}(\hat x) 
|\text{det} J_{\mu,T}(\hat x)| \,dx
\end{split}
\end{equation*}
Again contributions for all test functions can be collected in a vector
\begin{equation*}
\begin{split}
(\mathcal{A}_T^V(R_T z))_m &=
\sum_{n} (z)_{g_T(n)} \int_{\hat T} (J_{\mu,T}^{-1}(\hat x) \hat\nabla p_n^{\hat T}(\hat x) )
\cdot (J_{\mu,T}^{-1}(\hat x) \hat\nabla p_m^{\hat T}(\hat x) ) |\text{det} J_{\mu,T}(\hat x)| \,dx\\
&\hspace{30mm}+ \int_{\hat T} q\left( \sum_n (z)_{g_T(n)} p_n^{\hat T}(\hat x) \right) p_m^{\hat T}(\hat x) 
|\text{det} J_{\mu,T}(\hat x)| \,dx
\end{split}
\end{equation*}


\subsubsection*{Putting it all together}

Now with these definitions in place the evaluation of the algebraic residual is
\begin{equation*}
R(z) = 
\sum_{T\in\mathcal{T}_h} R_T^T \mathcal{A}_T^V(R_T z)
  + \sum_{T\in\mathcal{T}_h} R_T^T \mathcal{L}_T^V
 + \sum_{F\in\mathcal{F}_h^{\partial\Omega}\cap\Gamma_N} R_T^T \mathcal{L}_F^B
\end{equation*}

The Jacobian of the residual is
\begin{equation*}
(J(z))_{i,j} = \frac{\partial R_i}{\partial z_j} (z) =
\sum_{(T,m,n) : (T,m)\in C(i) \wedge (T,n)\in C(j)} \frac{\partial (\mathcal{A}_T^V)_m}{\partial z_n} .
(R_T z)
\end{equation*}
Note that:
\begin{enumerate}[a)]
\item Entries of the Jacobian can be computed element by element.
\item The derivative is independent of the $\lambda$-terms as
they only depend on the test functions.
\item In the implementation below the Jacobian is computed numerically
by finite differences. This can be achieved automatically by deriving from an
additional base class.
\end{enumerate}

\section{Realization in PDELab}

Need to explain the parameter class

% bibtex bibliography
\bibliographystyle{plain}
\bibliography{tutorial01.bib}

\end{document}
