\documentclass[12pt,a4paper]{article}

\usepackage[utf8]{inputenc}
\usepackage[a4paper,total={150mm,240mm}]{geometry}
\usepackage[american]{babel}

\usepackage{float}
\usepackage{babel}
\usepackage{amsmath}
\usepackage{tikz}
\usepackage{graphicx}
\usepackage{amssymb}

\usepackage{listings}
\definecolor{listingbg}{gray}{0.95}
\lstset{language=C++,basicstyle=\ttfamily\small,frame=single,backgroundcolor=\color{listingbg}}
% \lstset{language=C++, basicstyle=\ttfamily,
%   keywordstyle=\color{black}\bfseries, tabsize=4, stringstyle=\ttfamily,
%   commentstyle=\it, extendedchars=true, escapeinside={/*@}{@*/}}

\usepackage{exercise}

\title{\textbf{Exercises for the introduction to the Grid Interface}}
\exerciselabel{Exercise}

\begin{document}

\exerciseheader

\begin{Exercise}{Constructing Dune grids}

In the source directory, you should see (using \lstinline!ls!) the source file \lstinline!grid_exercise1.cc!,
which is the relevant code for this exercise. You build it like this:
\begin{lstlisting}
  [user@dunevm grid-exercise]$ make
\end{lstlisting}

Running \lstinline!make! will build your application (in this case: the exercise). The resulting output
is an executable, which in this cases is called \lstinline!exercise1!. You can execute it like this:

\begin{lstlisting}
  [user@dunevm grid-exercise]$ ./exercise1
\end{lstlisting}


Open the file \texttt{exercise1.cc} in a text editor.  It is an
example code that creates a structured grid (using the DUNE class
\texttt{YaspGrid}). The printgrid function then visualizes the grid
as a png file with some useful information, like global and local
indices, boundary intersections and such. You can have a look at the
visualization after a succesful run of the executbale \lstinline!exercise1!
with:

\begin{lstlisting}
  [user@dune-vm grid-exercise]$ ristretto printgrid_0.png
\end{lstlisting}

By default, a 4x4 grid is generated. You can change this to some
other number, recompile and rerun the executable and have a look at
the new grid visualization.

The code then iterates over all elements of this grid and
over all intersections of each element.  The code is meant to print
some information about the grid cells and the intersections (but it
does not yet).  The file is intermingled by suggestions what to print.
You are invited to follow these suggestions or to try any of the
member functions you learned about in the lectures.

\end{Exercise}

\end{document}
