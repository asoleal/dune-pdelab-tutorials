\documentclass[12pt,a4paper]{article}

\usepackage[utf8]{inputenc}
\usepackage[a4paper,total={150mm,240mm}]{geometry}
\usepackage[american]{babel}

\usepackage{float}
\usepackage{babel}
\usepackage{amsmath}
\usepackage{tikz}
\usepackage{graphicx}
\usepackage{amssymb}

\usepackage{listings}
\definecolor{listingbg}{gray}{0.95}
\lstset{language=C++,basicstyle=\ttfamily\small,frame=single,backgroundcolor=\color{listingbg}}
% \lstset{language=C++, basicstyle=\ttfamily,
%   keywordstyle=\color{black}\bfseries, tabsize=4, stringstyle=\ttfamily,
%   commentstyle=\it, extendedchars=true, escapeinside={/*@}{@*/}}

\usepackage{exercise}

\title{\textbf{Exercises for the introduction to the Grid Interface}}
\exerciselabel{Exercise}

\begin{document}

\exerciseheader

\begin{Exercise}{Constructing Dune grids}

In this exercise you should experiment with constructing different grids. The file
\lstinline!exercise_grid2.cc! contains a code, which
\begin{itemize}
 \item defines the \lstinline!GridType! to one of a list of available Dune grids
 \item constructs a grid either through one of the three basic factory concepts:
  \begin{itemize}
   \item \lstinline!StructuredgridFactory! for equidistant grids
   \item \lstinline!GmshReader! for unstructured grids
   \item \lstinline!TensorGridFactory! for tensor product grids
  \end{itemize}
 \item potentially refines the grid once globally (disabled by default)
 \item fills a data structure that maps each cell to its index in the index set.
 \item Outputs this data structure to a vtk file which can be visualized in \lstinline!paraview!
\end{itemize}

Try to construct as many different grids as possible and look at the result in paraview.
Here are some questions to guide your exploration of grid construction:
\begin{itemize}
 \item Find out (visually) how the elements in a \lstinline!YaspGrid! are ordered in the index set.
 \item Construct a structured grid with an unstructured grid manager
 \item Load an unstructured grid from one of the \lstinline!.msh! files you find in the exercise directory.
 \item Construct a \lstinline!YaspGrid! for the domain $[-1,1]^2$
 \item Enable the global refinement in the code and observe the effect on the index set for structured and unstructured grids.
 \item Build a tensor product \lstinline!YaspGrid! with and without global refinement. What do you observe?
\end{itemize}

\end{Exercise}

\end{document}
