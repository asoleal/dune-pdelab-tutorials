\documentclass[12pt,a4paper]{article}

\usepackage[utf8]{inputenc}
\usepackage[a4paper,total={150mm,240mm}]{geometry}
\usepackage[american]{babel}

\usepackage{hyperref}
\usepackage{amsmath}

\usepackage{xcolor}
\usepackage{listings}
\definecolor{listingbg}{gray}{0.95}
\lstset{
  language=c++,
  basicstyle=\ttfamily\small,
  frame=single,
  backgroundcolor=\color{listingbg},
  breaklines=true,
  postbreak=\raisebox{0ex}[0ex][0ex]{\ensuremath{\color{red}\hookrightarrow\space}}
}

\usepackage{todonotes}


\usepackage{exercise}
\title{\textbf{DUNE-PDELab-Tutorials Exercise Workflow}}
\subtitle{How to build and run exercises}
\exerciselabel{Exercise}

\begin{document}
\exerciseheader

You will learn here how to get and build all necessary dune-modules
for building the dune-pdelab-tutorials module.  Besides that you will
learn how to build and run the code from the tutorials and exercises.

First you need to install all the necessary Dune modules.  The
dependencies for installing DUNE can be found in the installation
notes
\footnote{\href{http://www.dune-project.org/doc/installation-notes.html}{http://www.dune-project.org/doc/installation-notes.html}}.
You will need the core modules
\footnote{\href{http://www.dune-project.org/download.html}{http://www.dune-project.org/download.html}}
and the dune-pdelab and dune-typetree modules
\footnote{\href{http://www.dune-project.org/pdelab/index.html}{http://www.dune-project.org/pdelab/index.html}}
for discretization.  If you are familiar with git you can instead
clone the git repositories
\footnote{\href{http://www.dune-project.org/downloadgit.html}{http://www.dune-project.org/downloadgit.html}}
\footnote{\href{http://www.dune-project.org/pdelab/index.html\#ToC14}{http://www.dune-project.org/pdelab/index.html\#ToC14}}
from the DUNE homepage.  Besides these dune-modules many examples need
ALUGrid or UGGrid for grid creation.  Dune-alugrid
\footnote{\href{https://gitlab.dune-project.org/extensions/dune-alugrid}{https://gitlab.dune-project.org/extensions/dune-alugrid}}
can be downlodaded like all other dune-modules.  UGGrid is not a
dune-module and you need to download and build it by yourself.
Download
\footnote{\href{http://www.dune-project.org/external\_libraries/install_ug.html}{http://www.dune-project.org/external\_libraries/install\_ug.html}}
the tarball and extract it next to your other dune-modules.  If you
enter a terminal and go to your dune-modules you should see the
following: \lstset{language=bash} % Lots of bash examples from now on
\begin{lstlisting}
[user@localhost]$ cd path/to/dune/folder
[user@localhost]$ ls
dune-alugrid
dune-common
dune-geometry
dune-grid
dune-istl
dune-localfunctions
dune-pdelab
dune-typetree
ug-3.12.1
ug-3.12.1.tar.gz
\end{lstlisting}
After that you need to following (we use ug version ug-3.12.1 as an
example):
\begin{lstlisting}
[user@localhost]$ mkdir external
[user@localhost]$ mkdir external/ug
[user@localhost]$ cd ug-3.12.1
[user@localhost]$ ./configure MPICC=$MPICC CC=$CXX --prefix=/absolute/path/to/dune/modules/external/ug CXXFLAGS="$CXXFLAGS" --enable-parallel --enable-dune && make $MAKE_FLAGS && make install
\end{lstlisting}
Here you have to set the absolute path to the \lstinline{ug} folder
you created before.

After downloading all dune-modules and installing the dependencies we
wan't to build the dune-pdelab-tutorials module.  In the following we
will describe a simple setup that should work in most cases.  If you
have problems you should first refer to the installation instructions
referenced above and see if you can find help there.

Building your DUNE modules is done by a script from dune-common.  The
easiest way to specify  options for the build process is passing an
opts file to dune-common.  This way you can e.g. switch optimization
flags for the compiler.  Here are two simple otps files that can be
used for debugging and release build:

\lstinputlisting[caption="Opts file for debuging."]{debug.opts}
\lstinputlisting[caption="Opts file with optimization."]{release.opts}
You once again have to set the absolute path to your \lstinline{ug}
folder.  If you want to use a different compiler just change the
corresponding paths in your opts files.

If you save these two files as debug.opts and release.opts in your
dunefolder you can use dune-common to build your sources:
\begin{lstlisting}
[user@localhost]$ ./dune-common/bin/dunecontroll --opts=debug.opts --builddir=release-build --module=dune-pdelab-tutorials all
[user@localhost]$ ./dune/dune-common/bin/dunecontrol --opts=debug.opts -builddir=debug-build --module=dune-pdelab-tutorials all
\end{lstlisting}
Dune uses cmake as build system that generates out of source builds.
In our case we specified the two build directories
\lstinline{release-build} and \lstinline{debug-build}.  In these
directories the whole structure of your modules is reproduced with
your build targets instead of the sources.  In order to run the
code for tutorial00 use:
\begin{lstlisting}
[user@localhost]$ cd release-build/dune-pdelab-tutorials/tutorial00/exercise
[user@localhost]$ ./exercise00
\end{lstlisting}
\todo[inline]{Name anpassen} If you made some changes to exercise00
\todo{Name anpassen} you can simply type \lstinline{make} to rebuild
your executable.

\end{document}