\documentclass[a4paper,12pt]{article}

\usepackage[utf8]{inputenc}
\usepackage[T1]{fontenc}
\usepackage[a4paper,total={150mm,240mm}]{geometry}
\usepackage{amsmath}
\usepackage{amsfonts}
\usepackage{amsthm}
\usepackage{amscd}
\usepackage{grffile}
\usepackage{tikz} 
\usepackage{eurosym} 
\usepackage{graphicx}
\usepackage{color}
\usepackage{listings}
\lstset{language=C++, basicstyle=\ttfamily, 
  keywordstyle=\color{black}\bfseries, tabsize=4, stringstyle=\ttfamily,
  commentstyle=\itshape, extendedchars=true, escapeinside={/*@}{@*/}}
\usepackage{paralist}
\usepackage{curves}
\usepackage{calc}
\usepackage{picinpar}
\usepackage{enumerate}
\usepackage{algpseudocode}
\usepackage{bm}
\usepackage{multibib}
\usepackage{hyperref}
\usepackage{textcase}
\usepackage{nicefrac}

\definecolor{listingbg}{gray}{0.95}

\theoremstyle{definition}
\newtheorem{exm}{Example}

\title{DUNE PDELab ---
Mathematical Abstractions for the Numerical Solution of Partial Differential Equations}
\author{Peter Bastian\\
  Universität Heidelberg, \\
  Interdisziplinäres Zentrum für Wissenschaftliches Rechnen\\
  Im Neuenheimer Feld 368, D-69120 Heidelberg\\
  \url{Peter.Bastian@iwr.uni-heidelberg.de}
}
\date{\today}

\begin{document}

\maketitle

\begin{abstract}
This short note describes the mathematical abstractions underlying the
software framework DUNE/PDELab for the numerical solution of partial
differential equations. It starts with the basic building block, the
stationary nonlinear problem, discusses the special case of the linear
problem and then goes on to instationary problems and systems of partial
differential equations. It ends with a short discussion of two fundamental
approaches to parallel solution techniques.
\end{abstract}

\section{Introduction}

State-of-the-art numerical solvers for partial differential equations (PDEs)
have a variety of aspects, among them:
\begin{enumerate}[1)]
\item Stationary and instationary solvers.
\item Linear and nonlinear solvers.
\item Various types of meshes in $d$ space dimensions.
\item Error control and adaptive mesh refinement.
\item Sequential and parallel solvers (overlapping and nonoverlapping).
\item Matrix-based and matrix-free computations.
\item Low- and high-order methods.
\end{enumerate}
DUNE/PDELab was designed to allow the implementation of a large variety
of methods for problems from different application areas with the orthogonality
of concepts in mind. This means that the different aspects listed above
can be combined in an arbitrary way allowing huge flexibility.
Flexibility, however, comes at a price: it requires the problem to
be stated in its most abstract way to identify commonalities and differences. 
This little note describes
these mathematical abstractions seperate from implementation
aspects to highlight their interplay without cluttering the description
with unnecessary detail.

\section{Stationary Nonlinear Problems}\label{Sec:BasicBuildingBlock}

The basic building block in PDELab is the solution of 
a stationary, nonlinear partial differential equation.
The idea is to write the discretization scheme in a weighted residual formulation:
\begin{equation}
\text{Find $u_h\in U_h$ s.t.:} \quad r_h(u_h,v)=0 \quad \forall v\in V_h.
\label{Eq:BasicBuildingBlock}
\end{equation}
Here $U_h$ and $V_h$ are finite-dimensional function spaces of dimension $n$ and $m$
and $r_h: U_h\times V_h\to\mathbb{K}$ ($\mathbb{K}=\mathbb{R}$ or 
$\mathbb{C}$) is the residual form. The residual 
form is always linear in the second argument and it may be linear or nonlinear in the first
argument. The subscript $h$ on the residual form indicates that its evaluation may be mesh dependent.
It turns out that a large class of finite element and finite volume schemes
can be cast in this general form.

\begin{exm} \label{Exm:NonlinearPoisson}
Consider the nonlinear Poisson equation with
Dirichlet and Neumann boundary conditions:
\begin{align*}
-\Delta u + q(u) &= f &&\text{in $\Omega$},\\
u &= g &&\text{on $\Gamma_D\subseteq\partial\Omega$},\\
-\nabla u\cdot \nu &= j &&\text{on $\Gamma_N=\partial\Omega\setminus\Gamma_D$}.
\end{align*}
$\Omega\subset\mathbb{R}^d$ is a domain, $q:\mathbb{R}\to\mathbb{R}$ is a given, possibly
nonlinear function and $f: \Omega\to\mathbb{R}$ is the source term and
$\nu$ denotes the unit outer normal to the domain

The weak formulation of this problem reads
\begin{equation*}
\text{Find $u\in U$ s.t.:} \quad r^{\text{NLP}}(u,v)=0 \quad \forall v\in V,
\end{equation*}
with the continuous residual form
\begin{equation*}
r^{\text{NLP}}(u,v) = \int_\Omega \nabla u \cdot \nabla v + (q(u)-f)v\,dx + \int_{\Gamma_N} jv\,ds .
\end{equation*}
The function spaces $U$ and $V$ are subspaces of the Sobolev space $H^1(\Omega)$ given by
$U= \{v\in H^1(\Omega) \,:\, \text{``$v=g$'' on $\Gamma_D$}\}$
and $V= \{v\in H^1(\Omega) \,:\, \text{``$v=0$'' on $\Gamma_D$}\}$. An element $u\in U$ may
also be written as $u=u_g + w$ where $u_g\in H^1(\Omega)$ satisfies the Dirichlet boundary
conditions and $w\in V$. A shorthand notation for $U$ is then ``$U=u_g+V$''.

The conforming finite element method \cite{BrennerScott,Elman2005,Ern}
replaces the function spaces
$U$ and $V$ by finite-dimensional versions $U_h$ and $V_h$ and uses the same residual form.
\hfill$\Box$
\end{exm}

\subsection*{Solving Stationary Nonlinear Problems}

In order to solve the abstract problem \eqref{Eq:BasicBuildingBlock} we use the fact that
every finite-dimensional function space is spanned by a basis. So, assume that
\begin{equation*}
U_h=\text{span}\{\phi_1,\ldots,\phi_n\}, \quad V_h=\text{span}\{\psi_1,\ldots,\psi_n\} .
\end{equation*}
Expanding the solution $u_h=\sum_{j=1}^n (z)_j\phi_j$ in the basis and
hereby introducing the coefficient vector $z\in\mathbb{K}^n$ we can
reformulate the problem as
\begin{align*}
\text{Find $u_h\in U_h$ s.t.:} && r_h(u_h,v)&=0 && \forall v\in V_h\\
\Leftrightarrow{} && r_h\left(\sum_{j=1}^n (z)_j\phi_j,\psi_i\right) &= 0 &&\forall i=1,\ldots,m\\
\Leftrightarrow{} && R(z) = 0,
\end{align*}
where $R: \mathbb{K}^n \to \mathbb{K}^m$ given by 
$R_i(z) = r_h\left(\sum_{j=1}^n (z)_j\phi_j,\psi_i\right)$ is a nonlinear, vector-valued function.

The solution of the nonlinear algebraic equation $R(z)=0$ is typically computed
in an iterative fashion using e.g. a fixed-point iteration of the form
\begin{equation}
z^{(k+1)} = G(z^{(k)}) = z^{(k)} - W(z^{(k)}) R(z^{(k)}) .
\end{equation}
Here $W(z^{(k)})$ is a preconditioner matrix, e.g. in Newton's method one
has 
\begin{equation*}
W(z^{(k)}) = (J(z^{(k)}))^{-1} \quad \text{where $(J(z^{(k)}))_{i,j} = \frac{\partial R_i}{\partial z_j}
(z^{(k)})$}
\end{equation*}
(we now assumed that $n=m$ and that the Jacobian $J(z^{(k)})$ is invertible).
Newton's method requires the solution of the linear system $J(z^{(k)}) w = R(z^{(k)})$ in each
step which could be done using either direct or iterative methods.
The implementation of the fixed-point scheme requires the following
algorithmic building blocks:
\begin{enumerate}[i)]
\item residual evaluation $R(z)$,
\item Jacobian evaluation $J(z)$ (or an approximation of it),
\item matrix-free Jacobian application $J(z) w$ (or an approximation).
\end{enumerate}
Only one of the methods ii) and iii) is required depending on the chosen
solution procedure.

\subsection*{Constraints}

As illustrated in example \ref{Exm:NonlinearPoisson} the function spaces
involved are often subspaces of a larger function space. PDELab provides
a general approach to building a subspace of a given function space through
the application of constraints. 

In order to construct a subspace $\tilde{U}_h$ 
of $U_h = \text{span}\left\{\phi_j \,:\, j\in J_h=\{1,\ldots,n\}\right\}$ 
\begin{enumerate}[i)]
\item select a subset of indices $\tilde{J}_h\subset J_h$
\item and set $\tilde{U}_h = \text{span}\left\{\tilde\phi_j \,:\, j\in \tilde{J}_h\right\}$ 
where the new basis functions have the form 
\begin{equation*}
\tilde\phi_j = \phi_j + \sum_{l\in J_h\setminus\tilde{J}_h} (B)_{j,l} \phi_l \quad \forall j\in \tilde{J}_h.
\end{equation*}
\end{enumerate}
Thus, any subspace of $U_h$ is characterized by $C=(\tilde{J}_h,B)$.
This abstractions allows to represent Dirichlet conditions ($J_h\setminus\tilde{J}_h$
are the indices of the Dirichlet nodes and $B=0$), hanging nodes
($J_h\setminus\tilde{J}_h$ are the indices of hanging nodes and $B$ represents
the interpolation conditions) or even rigid body modes.

\subsection*{More Structure to the Residual Form}

So far the residual form was deliberately kept very abstract, except for
the condition on linearity with respect to the second argument.
Finite element and finite volume methods, however, are defined via
appropriate weak formulations which involve integrals where the domain
of integration naturally decomposes into subdomains and basis functions
have local support. 

To that end assume that the domain $\Omega$ is covered by a mesh
$\mathcal{T}_h = \{T_1, \ldots, T_M\}$ consisting of elements
which are closed point sets satisfying
\begin{equation}
\bigcup_{T\in \mathcal{T}_h} T = \overline{\Omega}, \quad 
\forall T, T' \in \mathcal{T}_h, T\neq T' : \mathring{T} \cap \mathring{T}' = \emptyset .
\end{equation}
The nonempty intersections $F = T_F^-\cap T_F^+$ 
of codimension 1 form the interior skeleton $\mathcal{F}_h^i=\{F_1,\ldots,F_N\}$.
Each intersection is equipped with a unit normal vector $\nu_F$ pointing from $T_F^-$ to $T_F^+$.
The intersections of an element $F=T_F^-\cap\partial\Omega$ with the domain
boundary form the set of boundary intersections $\mathcal{F}_h^{\partial\Omega}=
\{F_1,\ldots,F_L\}$. Each boundary intersection is equipped with a unit normal vector
$\nu_F$ which coincides with the unit outer normal to the domain. Finally, we define the restriction
of a function $u\in U$ to an element by
\begin{equation*}
(R_T u)(x) = u(x) \quad \forall x\in\mathring{T} .
\end{equation*}
Note that the restriction of a function to element $T$ is only defined in
the interior of $T$. On interior intersections $F$, functions may be two-valued
and appropriate limits from within the elements $T_F^-, T_F^+$ need to be defined.

With that notation in place PDELab assumes that the residual form has
the following general structure:
\begin{equation}
\begin{split}
r(u,v) &= 
\sum_{T\in\mathcal{T}_h} \alpha_T^V(R_T u, R_T v) 
+ \sum_{T\in\mathcal{T}_h} \lambda_T^V(R_T v) \\
&\qquad+ \sum_{F\in\mathcal{F}_h^i} \alpha_F^S(R_{T_F^-} u,R_{T_F^+} u, R_{T_F^-} v, R_{T_F^+} v)\\
&\qquad+ \sum_{F\in\mathcal{F}_h^{\partial\Omega}} \alpha_F^B(R_{T_F^-} u, R_{T_F^-} v)
+ \sum_{F\in\mathcal{F}_h^{\partial\Omega}} \lambda_F^B(R_{T_F^-} v) .
\end{split}\label{eq:GeneralResidualForm}
\end{equation}
The five terms comprise volume integrals (superscript $V$), interior skeleton integrals
(superscript $S$) and boundary integrals (superscript $B$). Furthermore, the
$\alpha$-terms depend on trial and test functions whereas the $\lambda$-terms only
depend on the test function and involve the data of the PDE.

\begin{exm} \label{Exm:NonlinearPoissonDetailedResidualForm}
The residual form from example \ref{Exm:NonlinearPoisson} can be cast into the
following form:
\begin{equation*}
r^{\text{NLP}}(u,v) = 
\sum_{T\in\mathcal{T}_h} \int_T \nabla u \cdot \nabla v + q(u)v \,dx
- \sum_{T\in\mathcal{T}_h} \int_T fv \,dx 
+ \sum_{F\in\mathcal{F}_h^{\partial\Omega}\cap\Gamma_N} \int_{F} jv\,ds
\end{equation*}
and involves $\alpha_T^V$, $\lambda_T^V$ and $\lambda_F^B$. \hfill$\Box$
\end{exm}

\begin{exm} \label{Exm:FiniteVolumeMethod}
As a another example consider the cell-centered finite volume method with two-point
flux approximation applied to the problem treated in example \ref{Exm:NonlinearPoisson}.
In this method the discrete function space involved is the space of piecewise constant
functions on the mesh
\begin{equation*}
W_h = \{w\in L^2(\Omega) \,:\,  \text{$w|_T=$ const for all $T\in\mathcal{T}_h$}\}
\end{equation*}
and the residual form reads:
\begin{equation*}
\begin{split}
r_h^{\text{CCFV}}(u_h,v) 
& = \sum_{T\in\mathcal{T}_h} q(u_h(x_T)) v(x_T) |T|
- \sum_{T\in\mathcal{T}_h} f(x_T) v(x_T) |T|\\
&\ - \sum_{F\in\mathcal{F}_h^i} 
\frac{u_h(x_{T_F^+})-u_h(x_{T_F^-})}{\|x_{T_F^+} - x_{T_F^-}\|}
\bigl(v(x_{T_F^-}) - v(x_{T_F^+})\bigr) |F|\\
&\ + \sum_{F\in\mathcal{F}_h^{\partial\Omega}\cap\Gamma_D} 
\frac{u_h(x_{T_F^-})}{\|x_{F} - x_{T_F^-}\|} v(x_{T_F^-}) |F| \\
&\ - \sum_{F\in\mathcal{F}_h^{\partial\Omega}\cap\Gamma_D} 
\frac{g(x_{F}))}{\|x_{F} - x_{T_F^-}\|} v(x_{T_F^-}) |F|
+ \sum_{F\in\mathcal{F}_h^{\partial\Omega}\cap\Gamma_N} j(x_{F}) v(x_{T_F^-}) |F| .
\end{split}
\end{equation*}
The cell-centered finite volume method in this form is applicable on
axi-parallel meshes and is based in applying Gauss' theorem, approximating
the derivative in normal direction by a finite difference (two-point
flux approximation) and approximating integrals by the mid-point rule
($x_T$, $x_F$ denotes the center point of an element or face and 
$|T|$, $|F|$ denote the measure of an element or face).
In this case all five different terms of the residual form are involved.
Note that both terms in the last line are part of $\lambda_F^B$ as
they do not involve the unknown function $u_h$. 
Also note that here no constraints on the function space are necessary.
Dirichlet as well as Neumann boundary conditions are built into the
residual form.\hfill$\Box$
\end{exm}

\subsection*{Linear Stationary Problems as a Special Case}

A linear stationary problem can be considered as a special case
of the nonlinear problem where the residual form reads
\begin{equation}
r(u_h,v) = a(u_h,v) -l(v)
\end{equation}
with a bilinear form $a_h$ and a linear form $l_h$.
The corresponding algebraic function
\begin{equation*}
R(z) = Az-b
\end{equation*}
is now affine linear, the matrix $A$ and vector $b$ are defined by
\begin{equation*}
(A)_{i,j} = a(\phi_j,\psi_i) \quad\text{and}\quad (b)_i = l(\psi_i)
\end{equation*}
and the Jacobian $J(z)=A$ is independent of $z$. Computing the solution
$u_h$ now amounts to solving the linear system $Az=b$.

\section{Instationary Problems}

PDELab assumes that an instationary (nonlinear) problem can be cast into the
general form
\begin{equation}
\text{Find $u_h(t)\in U_h$ s.t.:} 
\quad d_t m_h(u_h(t),v;t) + r_h(u_h(t),v;t) = 0 
\quad \forall v\in V_h
\label{Eq:Instationary}
\end{equation}
after semi-discretization in space (method of lines approach).
One way to arrive at a fully discrete scheme is now to 
apply the one-step-$\theta$ rule which reads:
\begin{equation}
\label{Eq:InstationaryProblem}
\begin{split}
\text{Find $u_h^{k+1}\in U_h$ s.t.:} 
\quad \frac{1}{\Delta t_k}&(m_h(u_h^{k+1},v;t^{k+1})-m_h(u_h^{k},v;t^k)) + \\
&\theta r_h(u_h^{k+1},v;t^{k+1}) + (1-\theta) r_h(u_h^{k},v;t^k) = 0 
\quad \forall v\in V_h.
\end{split}
\end{equation}
Here, $0=t^0 < t^1 < \ldots < t^K=t_F$ is a subdivision of the 
time interval $\Sigma=(0,t_F)$, $\Delta t_k = t^{k+1}-t^k$ is the time step
and $u_h^{k}$ is an approximation of $u_h(t^k)$. For $\theta=1$
we obtain the implicit Euler method, for $\theta=1/2$ the Crank-Nicolson method
and for $\theta=0$ the explicit Euler method. With $u_h^k$ given
and $u_h^{k+1}$ unknown, the left hand side of equation \eqref{Eq:InstationaryProblem}
can be considered a new residual form built up from a linear combination of the
temporal residual form $m_h(u_h,v;t)$ and the spatial residual form $r_h(u_h,v;t)$
evaluated at different points in time.
Various explicit and diagonally implicit Runge-Kutta \cite{alexander:77} 
methods result in variants of
equation \eqref{Eq:InstationaryProblem} which yield higher-order accuracy in
time under appropriate stability properties.

\begin{exm} \label{Exm:NonlinearHeatEquation}
As an example consider the nonlinear heat equation
\begin{align*}
\partial_t u -\Delta u + q(u) &= f &&\text{in $\Omega\times\Sigma$},\\
u &= g &&\text{on $\Gamma_D\subseteq\partial\Omega$},\\
-\nabla u\cdot \nu &= j &&\text{on $\Gamma_N=\partial\Omega\setminus\Gamma_D$},\\
u &= u_0 &&\text{at $t=0$}.
\end{align*}
Here, the parameter functions $f$, $g$, $j$ may also depend on time
(and, with some restrictions, the subdivision into Dirichlet and Neumann boundary
can be time-dependent). The initial
condition $u_0$ is a function of $x\in\Omega$. Then the residual forms 
involved in the conforming finite element method are
\begin{align*}
m^{\text{NLH}}(u,v) &= \int_\Omega u v \,dx, \\
r^{\text{NLH}}(u,v) &= \int_\Omega \nabla u \cdot \nabla v + (q(u)-f)v\,dx + \int_{\Gamma_N} jv\,ds .
\end{align*}
Note that the spatial residual form $r_h^{\text{NLH}}$ is the same as before
which allows for a very easy implementation. \hfill$\Box$
\end{exm}

All considerations of Section \ref{Sec:BasicBuildingBlock} apply here as well.
The nonlinear solver can be applied to the new residual form and for 
linear problems a linear algebraic problem is obtained in each time step.

\subsection*{Explicit Time Stepping Schemes}

Considering the case of the explicit Euler method in \eqref{Eq:InstationaryProblem}
results in
\begin{equation*}
\text{Find $u_h^{k+1}\in U_h$ s.t.:} 
\quad m_h(u_h^{k+1},v;t)-m_h(u_h^{k},v;t) +
\Delta t_k r_h(u_h^{k},v;t) = 0 
\quad \forall v\in V_h.
\end{equation*}
For certain spatial schemes, e.g. finite volume or discontinuous Galerkin,
and assuming $m_h$ to be bilinear, the corresponding algebraic system
to be solved is diagonal:
\begin{equation}
Dz^{k+1} = s^k - \Delta t^k q^k.
\end{equation}
Moreover, a stability condition restricting the time step $\Delta t^k$ 
has to be obeyed. The maximum allowable time step can be computed
explicitly for the simplest schemes but depends on the mesh $\mathcal{T}_h$.
For explicit time-stepping schemes therefore the following algorithm is employed:
\begin{enumerate}[i)]
\item While traversing the mesh assemble the vectors $s^k$ and
$q^k$ separately and compute the maximum time step $\Delta t^k$.
\item Form the right hand side $b^k=s^k - \Delta t^k q^k$ and ``solve'' the 
diagonal system $Dz^{k+1} = b^k$ (can be done in one step).
\end{enumerate}
This procedure can be applied also to more general time-stepping schemes
such as strong stability preserving Runge-Kutta methods \cite{shu:88}.

\section{Systems of Partial Differential Equations}

All the considerations above carry over to the case of systems of
partial differential equations when Cartesian products of
functions spaces are introduced, i.e. the abstract stationary problem then reads
\begin{equation}
\text{Find $u_h\in U_h=U_h^1\times \ldots \times U_h^s$ s.t.:} \quad r_h(u_h,v)=0 
\quad \forall v\in V_h=V_h^1\times\ldots\times V_h^s
\label{Eq:BasicSystemBuildingBlock}
\end{equation}
with $s$ the number of components in the system. Again the concepts
are completely orthogonal meaning that $r_h$ might be linear or nonlinear
in its first argument and the instationary case works as well as is shown
in the following example.

\begin{exm} \label{Exm:WaveEquation}
As an example for a system we consider the wave equation with reflective boundary conditions:
\begin{align*}
\partial_{tt} u-c^2\Delta u  &= 0 &&\text{in $\Omega\times\Sigma$},\\
u &= 0 &&\text{on $\partial\Omega$},\\
u &= q &&\text{at $t=0$},\\
\partial_t u &= w &&\text{at $t=0$}.
\end{align*}
Renaming $u_0=u$ and introducing $u_1=\partial_t u_0 =\partial_t u$ we can write the wave equation 
as a system of two equations:
\begin{align*}
\partial_t u_1 - c^2\Delta u_0 &=0 &&\text{in $\Omega\times\Sigma$},\\
\partial_t u_0 - u_1 &=0 &&\text{in $\Omega\times\Sigma$},\\
u_0 &= 0 &&\text{on $\partial\Omega$},\\
u_0 &= q &&\text{at $t=0$},\\
u_1 &= w &&\text{at $t=0$}.
\end{align*}
Note that there are no constraints on the function space for $u_1$
and neither are there boundary conditions for the equation $u_1=\partial_t u_0$.
We note that different formulations of the wave equation as a system are
possible. \cite{Eriksson} uses $\Delta u_1 = \Delta \partial_t u_0$ as 
an equation to define $u_1$ and the formulation as a first order hyperbolic
system in the equations of linear acoustics allows the use of appropriate
upwinding techniques \cite{LeVeque}. 

Multiplying the first equation with test function $v_0$ and the second with test function $v_1$
and using integration by parts we arrive at the weak formulation: Find $(u_0(t),u_1(t))\in
U_0\times U_1$ s.t.
\begin{equation*}
d_t \left[ (u_0,v_1)_{0,\Omega} + (u_1,v_0)_{0,\Omega}\right]
+ \left[ c^2 (\nabla u_0,\nabla v_0)_{0,\Omega} -(u_1,v_1)_{0,\Omega} \right] = 0 
\quad \forall (v_0,v_1)\in U_0\times U_1
\end{equation*}
where we readily identify the temporal and spatial residual forms:
\begin{align*}
m^{\text{WAVE}}((u_0,u_1),(v_0,v_1)) &= (u_0,v_1)_{0,\Omega} + (u_1,v_0)_{0,\Omega},\\
r^{\text{WAVE}}((u_0,u_1),(v_0,v_1)) &= c^2 (\nabla u_0,\nabla v_0)_{0,\Omega} - (u_1,v_1)_{0,\Omega} .
\end{align*}
Here we used the notation of the $L^2$ inner product $(u,v)_{0,\Omega} = \int_\Omega
u v \, dx$.
\hfill$\Box$
\end{exm}


\section{Solving in Parallel}

In this section we consider the solution of PDEs on parallel computers
from an implementation point of view.

One approach relies on an additive splitting of the function space $U_h$
into subspaces $U_{h,i}$:
\begin{equation}
U_h = U_{h,1} + \ldots  + U_{h,p}
\label{eq:SubspaceSplitting}
\end{equation}
The subspaces may overlap, i.e. the splitting of any function $u_h = u_{h,1} + \ldots + u_{h,p}$
is not unique. 

Given such a splitting of the trial space as well as the test space the additive subspace
correction method \cite{Xu:1992:IMS:146428.146431} computes a new iterate as
\begin{equation*}
u_h^{k+1} = u_h^k + \sum_{i=1}^p w_{h,i}, \quad r_h(u_h^k+w_{h,i},v_i) = 0 \qquad \forall 
v_i\in V_{h,i}.
\end{equation*}
A particular choice for the space splitting is the overlapping domain decomposition
method where $U_{h,i} = \text{span}\{\phi_j\,:\, j\in I_{h,i}\}$ and $I_h = \bigcup_{i=1}^p
I_{h,i}$ is a possibly nonoverlapping splitting of the index set. Finite element basis functions
usually have local support and the space decomposition gives rise to
an overlapping domain decomposition $\Omega = \bigcup_{i=1}^p \Omega_i$ where
$\Omega_i = \bigcup_{j\in I_{h,i}} \text{supp}(\phi_j)$. Even if the splitting of the
index set is nonoverlapping (which corresponds to a block Jacobi solver) the domain
decomposition is overlapping. In terms of the underlying matrices the subspace correction
approach corresponds to a rowwise decomposition of the matrix. 
An advantage of the overlapping formulation is that the evaluation of the 
residual \eqref{eq:GeneralResidualForm} can be performed without
communication when $u_h$ is known in each subdomain.

A second approach relies on an additive splitting of the residual form
\begin{equation}
r_h(u_h,v) = \sum_{i=1}^p r_{h,i}(u_h,v) .
\label{eq:ResidualFormSplitting}
\end{equation}
This is accomplished naturally by a nonoverlapping partitioning of the mesh:
\begin{equation*}
\mathcal{T}_h = \bigcup_{i=1}^p \mathcal{T}_{h,i}, \quad
\mathcal{T}_{h,i}\cap \mathcal{T}_{h,j} = \emptyset \qquad \forall i\neq j.
\end{equation*}
The corresponding subdomains $\overline{\Omega}_i= \bigcup_{T\in\mathcal{T}_{h,i}} T$
are then nonoverlapping. The evaluation of the skeleton terms
in the general residual form \eqref{eq:GeneralResidualForm} requires some additional thought
and can be either achieved through communication during evaluation of the residual 
or through the introduction of additional overlap solely for this purpose (ghost cells
in DUNE's data decomposition model). The splitting of the residual form \eqref{eq:ResidualFormSplitting}
naturally carries over to an additive splitting of the Jacobian matrix.

The residual form decomposition is the natural way
to go with nonoverlapping domain decomposition methods (or Schur complement methods)
such as balancing or FETI-DP \cite{TosWid05}. For finite volume and discontinuous Galerkin methods
the space decomposition and residual form decomposition approach lead
to very similar methods.

% bibtex bibliography
\bibliographystyle{plain}
\bibliography{abstractions.bib}

\end{document}
