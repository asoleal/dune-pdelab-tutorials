\documentclass[12pt,a4paper]{article}

\usepackage[utf8]{inputenc}
\usepackage[a4paper,total={150mm,240mm}]{geometry}
\usepackage[american]{babel}

\usepackage{float}
\usepackage{babel}
\usepackage{amsmath}
\usepackage{tikz}
\usepackage{graphicx}
\usepackage{amssymb}
\usepackage{wrapfig}
\usepackage{todonotes}
\usepackage{hyperref}

\usepackage{listings}
\definecolor{listingbg}{gray}{0.95}
\lstset{
  language=C++,
  basicstyle=\ttfamily\small,
  frame=single,
  backgroundcolor=\color{listingbg},
  breaklines=true,
  postbreak=\raisebox{0ex}[0ex][0ex]{\ensuremath{\color{red}\hookrightarrow\space}}
  }


\newcommand{\vx}{\vec x}
\newcommand{\grad}{\vec \nabla}
\newcommand{\wind}{\vec \beta}
\newcommand{\Laplace}{\Delta}
\newcommand{\mycomment}[1]{}


% Exercise stylesheet
\usepackage{exercise}

\title{\textbf{Exercises for Tutorial04}}
\subtitle{Finite Elements for the Wave  Equation}
\exerciselabel{Exercise}

\begin{document}

\exerciseheader

In this exercise you will:
\begin{itemize}
\item Work on the problem presented in \lstinline{tutorial04}.
\item Try various time integrators, in particular the Crank-Nicolson method.
\item Explore polynomial degrees greater than $2$ by changing the blocking
to \lstinline{none}.
\item Implement your own local operator that switches to an elliptic projection method (see the tutorial notes and Eriksson et al. \cite{Eriksson})
\item Check the energy conservation property in the numerical scheme.
\end{itemize}

\begin{Exercise}{Getting to Know the Code}

  \lstset{language=bash}

  The code of \textbf{exercise04} solves the wave equation formulated as a first order in time system. As already explained in \lstinline{tutorial04}, we can write the wave equation as a system of two equations by substituting $u_0=u$ and introducing $u_1=\partial_t u_0 =\partial_t u$:
\begin{subequations}
\label{eq:SystemForm1}
\begin{align}
\partial_t u_1 - c^2\Delta u_0 &=0 &&\text{in $\Omega\times\Sigma$}, \label{eq:2a}\\
\partial_t u_0 - u_1 &=0 &&\text{in $\Omega\times\Sigma$}, \label{eq:2b}\\
u_0 &= 0 &&\text{on $\partial\Omega$},\\
u_1 &= 0 &&\text{on $\partial\Omega$},\\
u_0 &= q &&\text{at $t=0$},\\
u_1 &= w &&\text{at $t=0$}.
\end{align}
\end{subequations}
Since $u_0=u=0$ on the boundary we also have $\partial_t u = u_1 = 0$ on the boundary.
Alternatively, one may also simply omit the boundary condition on $u_1$.


  The code forthis exercise can be recompiled individually \textbf{in
    your build directory} by typing make:
  \begin{lstlisting}
  [user@localhost]$ cd release-build/dune-pdelab-tutorials/tutorial04/exercise/src
  [user@localhost]$ make
  \end{lstlisting}

  The structure of the code is very similar to the previous tutorials and
  consists of the following files:
  \begin{itemize}
  \item \lstinline!exercise04.ini! -- holds parameters read by various parts of the code which control the execution,
  \item \lstinline!exercise04.cc! -- main program,
  \item \lstinline!driver.hh! -- instantiates the necessary PDELab classes for solving a linear instationary problem and finally solves the problem,
  \item \lstinline!wavefem.hh! -- contains the local operator classes \lstinline!WaveFEM! and \lstinline!WaveL2! realizing the spatial and temporal residual forms $r(u,v,t)$ and $m(u,v,t)$, respectively.
  \end{itemize}

  As in the previous exercises you can control most of the settings
  through the ini-file \lstinline!exercise04.ini!. Get an overview of
  the configurable settings, compile and run \lstinline!exercise04!.

  The program writes output with the extension \lstinline!pvd!. This is one
  of several ways to write VTK output for the instationary case,
  c.f.~the documentation of tutorial03. The \lstinline!pvd!-file can
  be visualized by ParaView and consists of a collection of the
  corresponding \lstinline!vtu!-files. In order to visualize a 1D solution, one can apply the ''Plot Over Line '' filter. Note that our solution is always given by $u_0$.

\end{Exercise}

\begin{Exercise}{ Try various time integrators, in particular the Crank-Nicolson method}

  We want to examine the numerical solution under different time
  discretization schemes, eg. Implicit Euler or Crank-Nicolson. In order to change the time discretization   scheme you will have to go to the file \lstinline!driver.hh! and
  search for the line
  \begin{lstlisting}
    Dune::PDELab::Alexander2Parameter<RF> pmethod;
  \end{lstlisting}

 Recall the previous \lstinline!exercise03! and change the one
 step $\theta$ scheme in a way that corresponds to the Crank-Nicolson method.

  \textbf{Note that} you can compare the solutions in ParaView. To do that you have to rename the second solution in \lstinline!exercise04.ini! and use the "Append Attributes" filter before "Plot Over Line". Do not forget to change the parameter \lstinline!torder! in \lstinline!exercise04.ini! in order to get the correct scheme.


\end{Exercise}


 \textbf{Remark} You can decide about the domain dimension, see \lstinline!exercise04.ini!. A 2D simulation will give you nicer pictures at the cost of longer run times.


\begin{Exercise}{ Explore polynomial degrees greater than $2$ by changing the blocking
to \lstinline{none}. }
 \textbf{Step 1:}
Find the place where your Local Finite Element Maps are created and make it possible to use polynomial degree 3

\begin{lstlisting}
Dune::PDELab::PkLocalFiniteElementMap<GV,DF,double,deg> FEM;
\end{lstlisting}

Compile your program and check the results.

\textbf{Note that} \lstinline!deg! is a static template parameter.

\textit{Program should compile and give a segmentation error when started.}

 \textbf{Step 2:}

Before we start please have a look at the description of the driver in \lstinline!tutorial04! and read the part that describes the specification of an ordering when creating
product spaces. Using fixed block structure in ISTL requires that the number of degrees of freedom per entity is constant for each geometry type. This is true for polynomial degrees one and two but not for higher polynomial degree!

 To avoid segfaults you need to change

  \begin{lstlisting}
  Dune::PDELab::istl::VectorBackend<Dune::PDELab::istl::Blocking::fixed> VBE;  \end{lstlisting}
 to
\begin{lstlisting}
  Dune::PDELab::istl::VectorBackend<Dune::PDELab::istl::Blocking::none> VBE;  \end{lstlisting}
 in  \lstinline!driver.hh!.

\end{Exercise}

\begin{Exercise}{Changing the Local Operator}
Now consider the elliptic projection as in Eriksson et al. \cite{Eriksson}, namely applying the Laplacian to equation \eqref{eq:2b}
\begin{equation}\label{eq:elip}
-\Delta \partial_t u_0 + \Delta u_1 = 0,
\end{equation}
which has the advantage of  energy conservation but requires additional smoothness properties.


The main work is now to change the local operators given in the
  file \lstinline{wavefem.hh}. This is done in several steps:
  \begin{itemize}
  \item Copy the file \lstinline{wavefem.hh} to a new file and rename it.
  \item Rename the local operators in the new file, e.g.~change \lstinline{waveL2} to \lstinline{WaveElip}.
  \item Include your new file in \lstinline{exercise04.cc} and change the types of \lstinline{LOP} and \lstinline{TLOP} in \lstinline{driver.hh}.
  \begin{lstlisting}

  // Make instationary grid operator
  double speedofsound = ptree.get("problem.speedofsound",1.0);
  typedef WaveFEM<FEM> LOP;
  LOP lop(speedofsound);
  typedef WaveL2<FEM> TLOP;
  TLOP tlop;
  \end{lstlisting}
\end{itemize}

After these preparations you need to implement the following change:
$$ \partial_t u_0 - u_1 = 0 \Rightarrow \Delta \partial_t u_0 - \Delta u_1 = 0.$$

\begin{itemize}

\item In the spatial local operator, change:
$$ - u_1 = 0 \Rightarrow -  \Delta u_1 = 0$$

See how it is done for the $\Delta u_0$, we recall the corresponding part of \lstinline{WaveFEM::alpha_volume()}:
  \begin{lstlisting}
// integrate both equations
RF factor = ip.weight() * geo.integrationElement(ip.position());
for (size_t i=0; i<lfsu0.size(); i++) {
	r.accumulate(lfsu0,i,c*c*(gradu0*gradphi[i][0])*factor);
	r.accumulate(lfsu1,i,-u1*phihat[i]*factor);
}
  \end{lstlisting}


\item In the temporal local operator, change:
 $$ \partial_t u_0  = 0 \Rightarrow \Delta \partial_t u_0= 0$$

See how it is done for the $\Delta u_0$, we recall the corresponding part of \lstinline{WaveL2::alpha_volume()}:
  \begin{lstlisting}
// integrate u*phi_i
for (size_t i=0; i<lfsu0.size(); i++) {
	r.accumulate(lfsu0,i,u1*phihat[i]*factor);
	r.accumulate(lfsu1,i,u0*phihat[i]*factor);
}
  \end{lstlisting}

\item Do not forget to update the \lstinline{jacobian_volume()} methods accordingly. As the problem is linear, this should not be too difficult.
\end{itemize}
\end{Exercise}
\begin{Exercise}{Energy conservation}
If we multiply \eqref{eq:2a} by $u_1$ and \eqref{eq:elip} by $u_0$ and add, the terms $-(\Delta u_0,u_1 )$ and $(\Delta u_1,u_0 )$ cancel out, leading to the conclusion that the energy $$E(t) = \|u_1\|^2 + \| \nabla u_0\|^2 $$
is constant in time.


Your task is to check the energy conservation for the elliptic projection and Crank-Nicolson in time. You can use the following PDELab utilities:
  \begin{lstlisting}
	Dune::PDELab::SqrGridFunctionAdapter
	Dune::PDELab::integrateGridFunction
	Dune::PDELab::DiscreteGridFunctionGradient
	Dune::PDELab::SqrGridFunctionAdapter
  \end{lstlisting}

If you have problems with this task check the online documentation at
\url{https://www.dune-project.org/doc/doxygen/html/modules.html} or the solution.
\end{Exercise}

\textbf{Additional task}\\
If you are done with these exercises, you can play with the initial conditions. Use the following setting in one space dimension:
\begin{itemize}
\item Change  the initial conditions to $\sin(2x)$ (implement it as a lambda function)
\item Change the domain size to $[0,\pi]$
\end{itemize}


\bibliographystyle{plain}
\bibliography{exercise04.bib}

\end{document}
\grid
