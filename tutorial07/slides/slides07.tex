%%%%%%%%%%%%%%%%%%%%%%%%%%%%%%%%%%%%%%%%%%%%%%%%%%%%%%%%%%%%%%%%%%%%%%%%%%%%%%%%
%%%%%%%%%%%%%%%%%%%%%%%%%%%%%%%%%%%%%%%%%%%%%%%%%%%%%%%%%%%%%%%%%%%%%%%%%%%%%%%%
%
% A general frame for lecture slides and lecture notes in one file
% using LaTeX beamer
%
%%%%%%%%%%%%%%%%%%%%%%%%%%%%%%%%%%%%%%%%%%%%%%%%%%%%%%%%%%%%%%%%%%%%%%%%%%%%%%%%
%%%%%%%%%%%%%%%%%%%%%%%%%%%%%%%%%%%%%%%%%%%%%%%%%%%%%%%%%%%%%%%%%%%%%%%%%%%%%%%%
\documentclass[ignorenonframetext,11pt]{beamer}
%\usepackage[ngerman]{babel}
%\usepackage[T1]{fontenc}
\usepackage[utf8]{inputenc}
\usepackage{lmodern}
\usepackage{amsmath,amssymb,amsfonts}


% only presentation 
\mode<presentation>
{
  \usetheme{default}
%  \usecolortheme{crane}
  \setbeamercovered{transparent}
%  \setlength{\parindent}{0pt}
%  \setlength{\parskip}{1.35ex plus 0.5ex minus 0.3ex}
%  \usefonttheme{structuresmallcapsserif}
  \usefonttheme{structurebold}
  \setbeamertemplate{theorems}[numbered]
  \usepackage{amscd}
}

% all after
\usepackage{tikz} 
\usepackage{pgfplots,adjustbox}
\usepackage{eurosym} 
\usepackage{graphicx}
\usepackage{multimedia}
\usepackage{psfrag}
\usepackage{listings}
\lstset{language=C++, basicstyle=\ttfamily, 
  keywordstyle=\color{black}\bfseries, tabsize=4, stringstyle=\ttfamily,
  commentstyle=\it, extendedchars=true, escapeinside={/*@}{@*/}}
\usepackage{curves}
%\usepackage{epic}
\usepackage{calc}
%\usepackage{picinpar}
%\usepackage{fancybox}
%\usepackage{xspace}
\usepackage{enumerate}
\usepackage{algpseudocode}
\usepackage{color}
\usepackage{bold-extra}
\usepackage{bm}
\usepackage{stmaryrd}
%\usepackage[squaren]{SIunits}
\usepackage{nicefrac}

\usepackage{fancyvrb,bbm,xspace}
\usepackage{lmodern}
\usepackage{fancyvrb,bbm,xspace}
\usepackage[binary-units]{siunitx}
\usepackage{xcolor,tabu}

\definecolor{niceblue}{rgb}{0.122,0.396,0.651}   %% 31, 101, 166 or #1F65A6
\definecolor{niceorange}{RGB}{255,205,86}        %% #FFCD56
\definecolor{nicered}{RGB}{220,20,60}                      %% rgb(220, 20, 60)
\definecolor{niceteal}{HTML}{00A9AB}
\definecolor{niceviolet}{HTML}{820080}

\definecolor{niceblueLight}{HTML}{91CAFB}
\definecolor{niceblueVeryLight}{HTML}{DDEFFF}

\usepackage{dsfont}

%\newcommand{\hlineabove}{\rule{0pt}{2.6ex}}
%\newcommand{\hlinebelow}{\rule[-1.2ex]{0pt}{0pt}}

%\usecolortheme[RGB={37,75,123}]{structure}
% \definecolor{structurecolor}{rgb}{0.905,0.318,0.071}

% \setbeamercolor{frametitle}{fg=black,bg=}
% \setbeamercolor{sidebar left}{fg=,bg=}

% \setbeamertemplate{headline}{\vskip4em}
% \setbeamersize{sidebar width left=.9cm}

% \setbeamertemplate{navigation symbols}{}
%\setbeamertemplate{blocks}[rounded][shadow=true]
%\setbeamertemplate{itemize items}[square]

\mode<presentation> 
{
\theoremstyle{definition}
}
\newtheorem{Def}{Definition}%[section]
\newtheorem{Exm}[Def]{Example}
\newtheorem{Lem}[Def]{Lemma}
\newtheorem{Rem}[Def]{Remark}
\newtheorem{Rul}[Def]{Rule}
\newtheorem{Thm}[Def]{Theorem}
\newtheorem{Cor}[Def]{Corollary}
\newtheorem{Obs}[Def]{Observation}
\newtheorem{Ass}[Def]{Assumption}
\newtheorem{Pro}[Def]{Property}
\newtheorem{Alg}[Def]{Algorithm}
\newtheorem{Prp}[Def]{Proposition}
\newtheorem{Lst}[Def]{Listing}

% Delete this, if you do not want the table of contents to pop up at
% the beginning of each subsection:
\AtBeginSection[]
{
  \begin{frame}<beamer>
    \frametitle{Contents}
    \tableofcontents[sectionstyle=show/shaded,subsectionstyle=hide/hide/hide]
%\tableofcontents[currentsection]
  \end{frame}
}

% Title definition
\mode<presentation>
{
  \title{DUNE PDELab Tutorial 07 - Overview\\
  {\small  Discontinuous Galerkin for Hyperbolic conservation laws }}
  \author{Linus Seelinger}
  \institute[]
  {
   Interdisziplinäres Zentrum für Wissenschaftliches Rechnen\\
   Im Neuenheimer Feld 205, D-69120 Heidelberg \\[6pt]
  }
  \date[\today]{\today}
}


% logo nach oben
\mode<presentation>
{
% No navigation symbols and no lower logo
\setbeamertemplate{sidebar right}{}

% logo
\newsavebox{\logobox}
\sbox{\logobox}{%
    \hskip\paperwidth%
    \rlap{%
      % putting the logo should not change the vertical possition
      \vbox to 0pt{%
        \vskip-\paperheight%
        \vskip0.35cm%
        \llap{\insertlogo\hskip0.1cm}%
        % avoid overfull \vbox messages
        \vss%
      }%
    }%
}

\addtobeamertemplate{footline}{}{%
    \usebox{\logobox}%
}
}

%%%%%%%%%%%%%%%%%%%%%%%%%%%%%%%%%%%%%%%%%%%%%%%%%%%%%%%%%%%%%%%%%%%%%%%%%%%%%%%%
%%%%%%%%%%%%%%%%%%%%%%%%%%%%%%%%%%%%%%%%%%%%%%%%%%%%%%%%%%%%%%%%%%%%%%%%%%%%%%%%
%
% now comes the individual stuff lecture by lecture
%
%%%%%%%%%%%%%%%%%%%%%%%%%%%%%%%%%%%%%%%%%%%%%%%%%%%%%%%%%%%%%%%%%%%%%%%%%%%%%%%%
%%%%%%%%%%%%%%%%%%%%%%%%%%%%%%%%%%%%%%%%%%%%%%%%%%%%%%%%%%%%%%%%%%%%%%%%%%%%%%%%

\begin{document}

\frame{\titlepage}

%%%%%%%%%%%%%%%%%%%%%%%%%%%%%%%%%%%%%%%%%%%%%%%%%%%%%%%%%%%%%%%%%%%%%%%%%%%%%%%%
%%%%%%%%%%%%%%%%%%%%%%%%%%%%%%%%%%%%%%%%%%%%%%%%%%%%%%%%%%%%%%%%%%%%%%%%%%%%%%%%

\begin{frame}
\frametitle{Motivation}
Discontinuous Galerkin methods offer
\begin{itemize}
  \item local mass conservation (Conforming FEM does not!)
  \item higher-order approximation (Finite Volume does not!)
\end{itemize}
\end{frame}

\begin{frame}
\begin{center}
\Large\textbf{First-Order Hyperbolic PDEs}
\end{center}
\end{frame}

\begin{frame}
\frametitle{Conservative Form}

\begin{equation*}
\label{eq:master_problem}
\partial_t u(x,t) + \nabla\cdot F(u(x,t),x,t) = g(u(x,t),x,t)  \quad\text{in $U=\Omega\times\Sigma$}
\end{equation*}
with initial conditions
\begin{equation*}
u(x,0) = u_0(x)
\end{equation*}

\begin{itemize}
  \item Spatial domain $\Omega=\mathbb{R}^{d}$, $d\in\mathbb{N}$, temporal domain $\Sigma=\mathbb{R}^+$
  \item $F(u,x,t)$ is called \textit{flux function}
  \item Arises naturally from conservation laws (e.g. mass, energy, ...)
\end{itemize}
\end{frame}



\begin{frame}
\frametitle{Quasi-Linear Form}

\begin{equation*}
\label{eq:master_nonconservative_form}
\partial_t u(x,t) + \sum_{j=1}^{d} B_j(u(x,t),x,t) \partial_{x_j} u(x,t) + \tilde{g}(u(x,t),x,t) = 0
\quad\text{in $\Omega\times\Sigma$}
\end{equation*}
with initial conditions
\begin{equation*}
u(x,0) = u_0(x)
\end{equation*}

\begin{itemize}
  \item Derived from conservative form via chain rule (needs smoothness of flux function)
\end{itemize}
\end{frame}

\begin{frame}
\frametitle{Hyperbolicity Criterion}

\begin{equation*}
\partial_t u(x,t) + \sum_{j=1}^{d} B_j(u(x,t),x,t) \partial_{x_j} u(x,t) + \tilde{g}(u(x,t),x,t) = 0
\quad\text{in $\Omega\times\Sigma$}
\end{equation*}

\begin{itemize}
  \item Is called \textit{hyperbolic} if
        \begin{equation*}\label{eq:BMatrix}
          B(u,x,t; y) = \sum_{j=1}^{d} y_j B_j(u,x,t)
        \end{equation*}
        is real diagonalizable (i.e. real eigenvalues, eigenvectors form basis)
  \item Property needed for theoretical and numerical treatment
\end{itemize}
\end{frame}

%%%%%%%%%%%%%%%%%%%%%%%%%%%%%%%%%%%%%%%%%%%%%%%%%%%%%%%%%%%%%%%%%%%%%%%%%%%%%%%%
%%%%%%%%%%%%%%%%%%%%%%%%%%%%%%%%%%%%%%%%%%%%%%%%%%%%%%%%%%%%%%%%%%%%%%%%%%%%%%%%

\begin{frame}
\begin{center}
\Large\textbf{Examples}
\end{center}
\end{frame}

\begin{frame}
\frametitle{Acoustic Wave Equation}

Linearized conservation laws:

\begin{align*}
  \partial_t \tilde{\rho} +  \nabla\cdot(\bar{\rho} \tilde{v}) &= 0 &&\text{(conservation of mass)}\\
  \partial_t (\bar\rho \tilde{v}) + \nabla \tilde{p} &= 0 &&\text{(conservation of momentum)}
\end{align*}

\begin{itemize}
  \item Describes propagation of acoustic waves through material
  \item Derived by linearizing mass/momentum conservation around background state
  \item Simplifying assumptions involved!
\end{itemize}
\end{frame}

\begin{frame}
\frametitle{Acoustic Wave Equation}
As hyperbolic system:

$$\partial_t u(x,t) + \nabla\cdot F(u(x,t),x,t) = 0,$$
where
$$u = \begin{pmatrix}
\varrho\\
q_1\\
\vdots\\
q_d
\end{pmatrix} ,\quad
F(u(x,t),x,t) = \left( \begin{matrix}
q_1  & q_2 & \dots & q_{d}\\
c^2\rho & 0 & \dots & 0\\
0 & c^2\rho & \dots & 0\\
\vdots & \vdots & \ddots & \vdots\\
0 & 0 & \dots & c^2\rho
\end{matrix} \right)\in \mathbb{R}^{m\times d} $$

\begin{itemize}
  \item Derived replacing momentum by $\tilde{q} = \bar{\rho} \tilde{v}$
\end{itemize}
\end{frame}


\begin{frame}
\frametitle{Shallow Water Equations}

\begin{equation*}\label{eq:swe2D}
\partial_t \begin{pmatrix}
h\\
u_1h\\
u_2h
\end{pmatrix}+ \nabla\cdot
\left( \begin{matrix}
u_1h   & u_2h\\
u_1^2h + \frac{1}{2}gh^2 & u_1u_2h\\
u_1u_2h & u_2^2h + \frac{1}{2}gh^2
\end{matrix}\right) = 0
\end{equation*}

\begin{itemize}
  \item Describes water height $h>0$ and velocity $u=(u_1,u_2)$
  \item Widely used for predictions of flooding, dam-breaks, tsunamis
  \item Nonlinear PDE, again derived from conservation law
\end{itemize}
\end{frame}

%%%%%%%%%%%%%%%%%%%%%%%%%%%%%%%%%%%%%%%%%%%%%%%%%%%%%%%%%%%%%%%%%%%%%%%%%%%%%%%%
%%%%%%%%%%%%%%%%%%%%%%%%%%%%%%%%%%%%%%%%%%%%%%%%%%%%%%%%%%%%%%%%%%%%%%%%%%%%%%%%

\begin{frame}
\begin{center}
\Large\textbf{Discontinuous Galerkin Methods}
\end{center}
\end{frame}

\begin{frame}
\frametitle{Finite Element Space}

\begin{equation*}
V_h^q = \left\{ v\in L^2(\Omega) \,:\,
v|_e = p\circ\mu_e^{-1}, p\in\mathbb{P}^{q,d}\right\}
\end{equation*}

\begin{itemize}
  \item $\mu_e : \hat{E} \to e$ maps reference element to mesh element
  \item $p\in\mathbb{P}^{q,d}$ polynomial space on ref. element
  \item Note: No continuity required $\rightarrow$ nonconforming method!
\end{itemize}
\end{frame}

\begin{frame}
\frametitle{Discretization}
For piecewise smooth test function $v$:

\begin{equation*}\label{eq:DG_identity}
\begin{split}
\int_\Omega \biggl[&\partial_t u + \sum_{j=1}^{d}\partial_{x_j}F_j(u,x,t)\biggr]\cdot v \,dx = \\
&= d_t (u,v)_\Omega - \sum_{e\in\mathcal{E}_h} \int_e F(u,x,t) : \nabla v\,dx\\
&\qquad + \sum_{f\in\mathcal{F}_h^i} \int_f \llbracket (F(u,s,t)n)\cdot v \rrbracket \,ds
+ \sum_{f\in\mathcal{F}_h^{\partial\Omega}} \int_f (F(u,s,t)n)\cdot v \,ds \,.
\end{split}
\end{equation*}

\begin{itemize}
  \item $\llbracket v \rrbracket (x) = v^-(x)-v^+(x)$ denotes jump across interface
  \item Need single-valued approximation of $F \cdot n$
\end{itemize}
\end{frame}

\begin{frame}
\frametitle{Numerical Fluxes}


\begin{align*}
\Phi :  \mathbb{R}^d \times U \times U \rightarrow \mathbb{R},
\end{align*}

approximating $F \cdot n$ on interfaces, fulfilling

\begin{itemize}
  \item Consistency: Approximation exact if solution locally continuous
  \item Conservation: $\Phi(n_1,u_1,u_2) + \Phi(n_2, u_2, u_1) = 0$

\end{itemize}
\end{frame}

%%%%%%%%%%%%%%%%%%%%%%%%%%%%%%%%%%%%%%%%%%%%%%%%%%%%%%%%%%%%%%%%%%%%%%%%%%%%%%%%
%%%%%%%%%%%%%%%%%%%%%%%%%%%%%%%%%%%%%%%%%%%%%%%%%%%%%%%%%%%%%%%%%%%%%%%%%%%%%%%%

\begin{frame}
\begin{center}
\Large\textbf{Implementation in DUNE}
\end{center}
\end{frame}

\begin{frame}
\frametitle{Implementation in DUNE}

\begin{itemize}
  \item Define particular numerical flux
  \item Define DG method's spatial and temporal local operators
  \item Use a \texttt{DGLocalFiniteElementMap}
\end{itemize}
Otherwise, same as any other instationary method!
\end{frame}

\end{document}

