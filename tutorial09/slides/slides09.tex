%%%%%%%%%%%%%%%%%%%%%%%%%%%%%%%%%%%%%%%%%%%%%%%%%%%%%%%%%%%%%%%%%%%%%%%%%%%%%%%%
%%%%%%%%%%%%%%%%%%%%%%%%%%%%%%%%%%%%%%%%%%%%%%%%%%%%%%%%%%%%%%%%%%%%%%%%%%%%%%%%
%
% A general frame for lecture slides and lecture notes in one file
% using LaTeX beamer
%
%%%%%%%%%%%%%%%%%%%%%%%%%%%%%%%%%%%%%%%%%%%%%%%%%%%%%%%%%%%%%%%%%%%%%%%%%%%%%%%%
%%%%%%%%%%%%%%%%%%%%%%%%%%%%%%%%%%%%%%%%%%%%%%%%%%%%%%%%%%%%%%%%%%%%%%%%%%%%%%%%
\documentclass[ignorenonframetext,11pt]{beamer}
%\usepackage[ngerman]{babel}
%\usepackage[T1]{fontenc}
\usepackage[utf8]{inputenc}
\usepackage{lmodern}
\usepackage{amsmath,amssymb,amsfonts}


% only presentation
\mode<presentation>
{
  \usetheme{default}
%  \usecolortheme{crane}
  \setbeamercovered{transparent}
%  \setlength{\parindent}{0pt}
%  \setlength{\parskip}{1.35ex plus 0.5ex minus 0.3ex}
%  \usefonttheme{structuresmallcapsserif}
  \usefonttheme{structurebold}
  \setbeamertemplate{theorems}[numbered]
  \usepackage{amscd}
}

% all after
\usepackage{tikz}
\usepackage{pgfplots,adjustbox}
\usepackage{eurosym}
\usepackage{graphicx}
\usepackage{multimedia}
\usepackage{psfrag}
\usepackage{listings}
\lstset{language=C++, basicstyle=\ttfamily,
  keywordstyle=\color{black}\bfseries, tabsize=4, stringstyle=\ttfamily,
  commentstyle=\it, extendedchars=true, escapeinside={/*@}{@*/}}
\usepackage{minted}
\usepackage{curves}
%\usepackage{epic}
\usepackage{calc}
%\usepackage{picinpar}
%\usepackage{fancybox}
%\usepackage{xspace}
\usepackage{enumerate}
\usepackage{algpseudocode}
\usepackage{color}
\usepackage{bold-extra}
\usepackage{bm}
\usepackage{stmaryrd}
%\usepackage[squaren]{SIunits}
\usepackage{nicefrac}

\usepackage{fancyvrb,bbm,xspace}
\usepackage{lmodern}
\usepackage{fancyvrb,bbm,xspace}
\usepackage[binary-units]{siunitx}
\usepackage{xcolor,tabu}

\definecolor{niceblue}{rgb}{0.122,0.396,0.651}   %% 31, 101, 166 or #1F65A6
\definecolor{niceorange}{RGB}{255,205,86}        %% #FFCD56
\definecolor{nicered}{RGB}{220,20,60}                      %% rgb(220, 20, 60)
\definecolor{niceteal}{HTML}{00A9AB}
\definecolor{niceviolet}{HTML}{820080}

\definecolor{niceblueLight}{HTML}{91CAFB}
\definecolor{niceblueVeryLight}{HTML}{DDEFFF}

\usepackage{dsfont}

%\newcommand{\hlineabove}{\rule{0pt}{2.6ex}}
%\newcommand{\hlinebelow}{\rule[-1.2ex]{0pt}{0pt}}

%\usecolortheme[RGB={37,75,123}]{structure}
% \definecolor{structurecolor}{rgb}{0.905,0.318,0.071}

% \setbeamercolor{frametitle}{fg=black,bg=}
% \setbeamercolor{sidebar left}{fg=,bg=}

% \setbeamertemplate{headline}{\vskip4em}
% \setbeamersize{sidebar width left=.9cm}

% \setbeamertemplate{navigation symbols}{}
%\setbeamertemplate{blocks}[rounded][shadow=true]
%\setbeamertemplate{itemize items}[square]

\mode<presentation>
{
\theoremstyle{definition}
}
\newtheorem{Def}{Definition}%[section]
\newtheorem{Exm}[Def]{Example}
\newtheorem{Lem}[Def]{Lemma}
\newtheorem{Rem}[Def]{Remark}
\newtheorem{Rul}[Def]{Rule}
\newtheorem{Thm}[Def]{Theorem}
\newtheorem{Cor}[Def]{Corollary}
\newtheorem{Obs}[Def]{Observation}
\newtheorem{Ass}[Def]{Assumption}
\newtheorem{Pro}[Def]{Property}
\newtheorem{Alg}[Def]{Algorithm}
\newtheorem{Prp}[Def]{Proposition}
\newtheorem{Lst}[Def]{Listing}


%% Typesetting C++
\def\CC{{C\nolinebreak[4]\hspace{-.05em}\raisebox{.4ex}{\tiny\bf ++}}}

% Delete this, if you do not want the table of contents to pop up at
% the beginning of each subsection:
\AtBeginSection[]
{
  \begin{frame}<beamer>
    \frametitle{Contents}
    \tableofcontents[sectionstyle=show/shaded,subsectionstyle=hide/hide/hide]
%\tableofcontents[currentsection]
  \end{frame}
}

% Title definition
\mode<presentation>
{
  \title{DUNE PDELab Tutorial 09\\
  {\small  Using code generation to create local operators}}
  \author{PDELab Team}
  \institute[]
  {
   Interdisziplinäres Zentrum für Wissenschaftliches Rechnen\\
   Im Neuenheimer Feld 205, D-69120 Heidelberg \\[6pt]
  }
  \date[\today]{\today}
}


% logo nach oben
\mode<presentation>
{
% No navigation symbols and no lower logo
\setbeamertemplate{sidebar right}{}

% logo
\newsavebox{\logobox}
\sbox{\logobox}{%
    \hskip\paperwidth%
    \rlap{%
      % putting the logo should not change the vertical possition
      \vbox to 0pt{%
        \vskip-\paperheight%
        \vskip0.35cm%
        \llap{\insertlogo\hskip0.1cm}%
        % avoid overfull \vbox messages
        \vss%
      }%
    }%
}

\addtobeamertemplate{footline}{}{%
    \usebox{\logobox}%
}
}

%%%%%%%%%%%%%%%%%%%%%%%%%%%%%%%%%%%%%%%%%%%%%%%%%%%%%%%%%%%%%%%%%%%%%%%%%%%%%%%%
%%%%%%%%%%%%%%%%%%%%%%%%%%%%%%%%%%%%%%%%%%%%%%%%%%%%%%%%%%%%%%%%%%%%%%%%%%%%%%%%
%
% now comes the individual stuff lecture by lecture
%
%%%%%%%%%%%%%%%%%%%%%%%%%%%%%%%%%%%%%%%%%%%%%%%%%%%%%%%%%%%%%%%%%%%%%%%%%%%%%%%%
%%%%%%%%%%%%%%%%%%%%%%%%%%%%%%%%%%%%%%%%%%%%%%%%%%%%%%%%%%%%%%%%%%%%%%%%%%%%%%%%

\begin{document}

\frame{\titlepage}

%%%%%%%%%%%%%%%%%%%%%%%%%%%%%%%%%%%%%%%%%%%%%%%%%%%%%%%%%%%%%%%%%%%%%%%%%%%%%%%%
%%%%%%%%%%%%%%%%%%%%%%%%%%%%%%%%%%%%%%%%%%%%%%%%%%%%%%%%%%%%%%%%%%%%%%%%%%%%%%%%
\begin{frame}
  \frametitle{Introduction}

  \lstinline{dune-codegen}, UFL, Workflow, Goal, Interface to dune-pdelab
\end{frame}

\begin{frame}
  \frametitle{Poisson Problem from Tutorial00}

  \begin{subequations}
    Strong formulation:
    \begin{align*}
      -\Delta u & = f \qquad\text{in $\Omega$}, \\
      u &= g \qquad\text{on $\partial\Omega$},
    \end{align*}

    Weak formulation: Find $u_h \in U_h$ with
    \begin{equation*}
      r_h^{Poisson}(u_h, v_h) = \int_\Omega \nabla u_h \cdot \nabla v_h \, dx
      - \int_\Omega f \, v_h \, dx = 0 \qquad \forall v_h \in V_h
    \end{equation*}

    Parameter functions:
    \begin{align*}
      f(x) = -2d \\
      g(x) = \| x \|_2^2
    \end{align*}
  \end{subequations}
\end{frame}

\begin{frame}
  \frametitle{UFL Formulation for the Poisson Problem}
  % \inputminted[fontsize=\scriptsize]{python}{../src/poisson.ufl}
  \lstinputlisting[basicstyle=\scriptsize]{../src/poisson.ufl}
\end{frame}

\begin{frame}
  \frametitle{Nonlinear Poisson from Tutorial01}

    Strong formulation:
    \begin{align*}
      -\Delta u + q(u) & = f \qquad\text{in $\Omega$}, \\
      u &= g \qquad\text{on $\partial\Omega$}, \\
      -\nabla u \cdot \mu = j
    \end{align*}

    Weak formulation: Find $u_h \in U_h$ with
    \begin{align*}
      r_h^{NLP}(u_h, v_h) & =
      \int_\Omega \nabla u_h \cdot \nabla v_h \, dx
      + \int_\Omega q(u) \, v \, dx \\
      &\quad - \int_\Omega f \, v_h \, dx
      + \int_{\Gamma_N} j \, v \, ds
      = 0 \qquad \forall v_h \in V_h
    \end{align*}

    Parameter functions:
    \begin{align*}
      f(x) = -2d \\
      g(x) = \| x \|_2^2
    \end{align*}
\end{frame}

\begin{frame}
  \frametitle{UFL: Adding the Nonlinearity}
  \begin{itemize}
  \item Adjust parameter functions $f$ and $g$
  \item Define nonlinearity
  \item Add nonlinearity to residual
  \end{itemize}

  \vfill

  \lstinputlisting[basicstyle=\scriptsize,
  linerange={6-15}]{../src/nonlinear_poisson.ufl}
\end{frame}

\begin{frame}
  \frametitle{UFL: Using Nitsche Boundary Condition}
  \begin{itemize}
  \item Add Nitsche boundary implementation
  \item Remove \lstinline{is_dirichlet} part, since the boundary condition is not built into the ansatz space.
  \end{itemize}

  \vfill

  \lstinputlisting[basicstyle=\scriptsize,
  linerange={18-26}]{../src/nonlinear_poisson_nitschebc.ufl}
\end{frame}

\begin{frame}
  \frametitle{Something about instationary?}
\end{frame}

\begin{frame}
  \frametitle{UFL language}
  jump, avg, normal,
\end{frame}

\begin{frame}[fragile]
  \frametitle{CMake: \lstinline{dune_add_generated_executable}}

  \begin{itemize}
  \item We need to generate \CC\ code and compile it
  \item Add a code generation target to your \lstinline{CMakeLists.txt}
    \inputminted[fontsize=\scriptsize]{cmake}{generated_executable.txt}
  \item \lstinline{UFLFILE}: UFL file describing the PDE
  \item \lstinline{INIFILE}: Ini file with code generation option under \lstinline{[formcompiler]} section
  \item \lstinline{TARGET}: Name of the executable
  \item \lstinline{SOURCE}: \CC\ file used for building the target. This is
    optional, if omitted a minimal driver willl be generated
  \end{itemize}
\end{frame}

\begin{frame}[fragile]
  \frametitle{CMake: \lstinline{dune_add_generated_executable}}

  \begin{itemize}
  \item Automated driver generation is mainly developed for software tests
  \item For complicated applications handwritten drivers will be
    necessary. This requires control over the file- and classname of the
    generated local operator.
  \item Can be done in the ini file
    \inputminted[fontsize=\scriptsize]{ini}{classname_filename.ini}
  \end{itemize}
\end{frame}

\begin{frame}[fragile]
  \frametitle{Options?}

  Global Options:

  \begin{lstlisting}
  explicit_time_stepping
  exact_solution_expression
  compare_l2errorsquared
  \end{lstlisting}

  Form Options:

  \begin{lstlisting}
  filename
  classname
  ? numerical_jacobian
  quadrature_order
  geometry_mixins
  ? enable_volume
  ? enable_skeleton
  ? enable_boundary
  \end{lstlisting}
\end{frame}

\begin{frame}[fragile]
  \frametitle{CMake: \lstinline{dune_add_generated_executable}}

  Examples?

  \inputminted[fontsize=\scriptsize, firstline=1, lastline=4]{cmake}{../src/CMakeLists.txt}
\end{frame}

\end{document}
