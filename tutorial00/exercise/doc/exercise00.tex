\documentclass[12pt,a4paper]{article}

\usepackage[utf8]{inputenc}
\usepackage[a4paper,total={150mm,240mm}]{geometry}
\usepackage[american]{babel}

\usepackage{float}
\usepackage{babel}
\usepackage{amsmath}
\usepackage{tikz}
\usepackage{graphicx}
\usepackage{amssymb}
\usepackage{todonotes}


\usepackage{listings}
\definecolor{listingbg}{gray}{0.95}
\lstset{
  language=c++,
  basicstyle=\ttfamily\small,
  frame=single,
  backgroundcolor=\color{listingbg},
  breaklines=true,
  postbreak=\raisebox{0ex}[0ex][0ex]{\ensuremath{\color{red}\hookrightarrow\space}}
}


\newcommand{\vx}{\vec x}
\newcommand{\grad}{\vec \nabla}
\newcommand{\wind}{\vec \beta}
\newcommand{\Laplace}{\Delta}
\newcommand{\mycomment}[1]{}


% Exercise stylesheet
\usepackage{exercise}

\title{\textbf{Exercises for Tutorial00}}
\subtitle{Poisson Equation}
\exerciselabel{Exercise}


\begin{document}

\exerciseheader

\begin{Exercise}{Poisson Equation with Piecewise Linear Elements}

In this exercise you will:
\begin{itemize}
\item Work on the problem presented in the tutorial and see how the
  code works.
\item Modify the implementation to solve a slightly different problem.
\item Determine the convergence of the finite element method.
\end{itemize}

\begin{enumerate}
\item {\sc Warming up}:\\
  The code presented in this tutorial solves the following elliptic
  equation with Dirichlet boundary conditions
  \begin{align*}
    \begin{array}{rcll}
      -\Laplace u  & = & f & \text{ in } \Omega, \\
      u & = & g & \text{ on } \partial\Omega,
    \end{array}
  \end{align*}
  where
  \begin{align*}
    f(x) = -2d \quad\text{and}\quad  g(x) = \sum_{i=1}^d (x)_i^2.
  \end{align*}

  First of all check that $u(x) = \sum_{i=1}^d (x)_i^2$ is a solution
  for any dimension $d$.

  This program solves this equation for $d=2$. The executable can be
  found in the build directory of this exercise:
  \begin{lstlisting}
cd course/release-build/dune-pdelab-tutorials/tutorial00/exercise/task
  \end{lstlisting}
  You can run the program with \lstinline{./exercise00}.  Some
  parameters of the program are controlled by the ini file
  \lstinline{src_dir/tutorial00.ini}:
  \lstinputlisting[language=bash]{../task/tutorial00.ini}

  Here you may set the mesh refinement and the names of the input and
  output files. Run the program with different refinement
  levels. Visualize the solution in ParaView, use the Calculator
  filter to visualize the difference $|u-u_h|$ and determine the
  maximum error. Note that $u$ and $u_h$ are called \lstinline{exact}
  and \lstinline{fesol} in the parview output.

\item {\sc Solving a new problem}:\\
  Now consider
  \begin{align}
    f(x) =  -\sum_{i=1}^d 6(x)_i \quad\text{and}\quad  g(x) = \sum_{i=1}^d (x)_i^3
  \end{align}
  and check that $u(x)=\sum_{i=1}^d (x)_i^3$ solves the PDE. Implement
  the new $f$ and $g$, rebuild and rerun your program.

\item {\sc Analysis of finite element error}:\\
  Produce a sequence of output files for different levels of mesh
  refinements $0, 1, 2, \ldots$ with suitable output
  filenames. Visualize the error $|u-u_h|$ in Para\-View and determine
  the maximum error on each level.  Determine the convergence rate.

\item {\sc Use a different solver}:\\
  You may also try to exchange the iterative linear solver by
  replacing it with the following lines int the source file
  \lstinline{exercise00.cc}:
  \begin{lstlisting}
typedef Dune::PDELab::ISTLBackend_SEQ_BCGS_SSOR LS;
LS ls(5000,true);
  \end{lstlisting}
  Compare the number of iterations which is given by the following lines
  in the output (here we have 12 iterations):
  \begin{lstlisting}
=== CGSolver
12       1.9016e-09
=== rate=0.144679, T=0.02362, TIT=0.00196833, IT=12
  \end{lstlisting}
\end{enumerate}
\end{Exercise}


\begin{Exercise}{Extending the Local Operator}
Now consider the extended equation of the form
\begin{align}
    \begin{array}{rcll}
      -\nabla\cdot (k(x) \nabla u) +a(x) u  & = & f & \text{ in } \Omega, \\
      u & = & g & \text{ on } \partial\Omega,
    \end{array}
\end{align}
with scalar functions $k(x)$, $a(x)$.

\begin{enumerate}
\item In a first step show that the weak formulation of the problem
  involves the new bilinear form
  \begin{align}
    a(u,v) = \int_\Omega k(x) \nabla u(x) \cdot \nabla v(x) + a(x) u(x) v(x) \,dx .
  \end{align}
\item The main work is to extend the local operator given by the class
  \lstinline{PoissonP1}. This can be done in several steps:
  \begin{itemize}
  \item Copy the class \lstinline{PoissonP1} to a new file and rename it.
  \item Provide analytic functions for $k(x)$ and $a(x)$ and pass
    grid functions to the local operator like it is done for $f(x)$.
  \item Extend the local operator to first handle $k(x)$ by assuming the
    function $k(x)$ to be \emph{constant} on mesh elements.
  \item Now extend the local operator to handle $a(x) u(x)$. Also assume
    the function $a(x)$ to be \emph{constant} on mesh elements.
  \end{itemize}
\item Plug in your new local operator in the driver code and test it.
  In the solution the code is tested for $k(x)=2(x)_1$,
  $a(x)=\sum_{i=1}^{d}(x)_i$ and $g=\sum_{i=1}^d (x)_i^3$ by choosing
  the right hand side $f(x)$ in such a way that $u(x)=\sum_{i=1}^d
  (x)_i^3$ solves the PDE.
\end{enumerate}
\end{Exercise}


\end{document}
